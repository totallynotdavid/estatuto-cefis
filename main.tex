\documentclass[11pt]{article}

\usepackage[record=off, style=list, toc, section, symbols, automake, abbreviations]{glossaries-extra}
\setabbreviationstyle[acronym]{long-short}
\makeglossaries%
\newabbreviation%
  [description={Centro de Estudiantes de Física, gremio estudiantil de la Escuela Profesional de Física de la UNMSM.}]
  {cefis}{CEFIS}{Centro de Estudiantes de Física}

\newabbreviation%
  [description={Universidad Nacional Mayor de San Marcos, la universidad a la que pertenece la E.P. de Física.}]
  {unmsm}{UNMSM}{Universidad Nacional Mayor de San Marcos}

\newabbreviation%
  [description={Escuela Profesional de Física, donde estudian los/as agremiados/as del CEFIS.}]
  {epf}{E.P. de Física}{Escuela Profesional de Física}

\newabbreviation%
  [description={Asamblea General de Estudiantes, máxima instancia deliberativa y decisoria del CEFIS.}]
  {age}{AGE}{Asamblea General de Estudiantes}

\newabbreviation%
  [description={Asamblea General Ordinaria, convocada dos veces por semestre académico regular.}]
  {ageo}{AGE-O}{Asamblea General Ordinaria}

\newabbreviation%
  [description={Asamblea General Extraordinaria, convocada cuando sea necesario para asuntos específicos.}]
  {agee}{AGE-E}{Asamblea General Extraordinaria}
  
\newabbreviation%
  [description={Junta Directiva del CEFIS, órgano ejecutivo y administrativo del CEFIS.}]
  {jd}{JD}{Junta Directiva}

\newabbreviation%
  [description={Junta de Representantes, órgano colegiado de coordinación de bases, fiscalización de la JD y representación estudiantil articulada.}]
  {jr}{JR}{Junta de Representantes}

\newabbreviation%
  [description={Asamblea de Base, instancia primaria de participación y deliberación por año académico de ingreso.}]
  {ab}{AB}{Asamblea de Base}

\newabbreviation%
  [description={Comité Electoral, órgano autónomo y temporal encargado de organizar y supervisar los procesos electorales.}]
  {ce}{CE}{Comité Electoral}

\newglossaryentry{estatuto}
{
  name={Estatuto},
  description={El presente documento normativo que rige la organización, funcionamiento, fines y principios del CEFIS.}
}

\newglossaryentry{agremiado}
{
  name={agremiado/a},
  plural={agremiados/as},
  description={Estudiante de pregrado con matrícula vigente en el semestre académico regular en la E.P. de Física, incluyendo aquellos con reserva de matrícula válida o en proceso de reactualización.}
}

\newglossaryentry{agremiado_habilitado}
{
  name={agremiado/a habilitado/a},
  plural={agremiados/as habilitados/as},
  description={Agremiado/a que cumple con la condición de tal y no tiene suspendidos sus derechos conforme al presente Estatuto.}
}

\newglossaryentry{registro_central}
{
  name={Registro Central del CEFIS},
  description={Archivo organizado, físico y/o digital, de toda la documentación oficial del CEFIS, administrado por la Secretaría de Actas y Economía.}
}

\newglossaryentry{base_activa}
{
  name={base activa},
  plural={bases activas},
  description={Cada uno de los cinco (5) años académicos de ingreso más recientes con estudiantes matriculados en el semestre vigente, representados en la JR.}
}

\newglossaryentry{delegatura}
{
  name={Delegatura},
  plural={Delegaturas},
  description={Cargo de representación de una Base Activa en la Junta de Representantes (JR), elegido por los miembros de dicha base.}
}

\newglossaryentry{subdelegatura}
{
  name={Subdelegatura},
  plural={Subdelegaturas},
  description={Cargo de representación suplente de una Base Activa en la Junta de Representantes (JR), elegido por los miembros de dicha base.}
}

\newglossaryentry{libro_contable}
{
  name={Libro Contable},
  description={Registro físico o digital donde se asientan todos los ingresos y egresos financieros del CEFIS de manera detallada y cronológica.}
}

\newglossaryentry{padron_electoral}
{
  name={padrón electoral},
  description={Lista oficial de agremiados/as habilitados/as para emitir su voto en los procesos electorales del CEFIS.}
}

\newglossaryentry{reglamento_asambleas}
{
  name={Reglamento de Asambleas},
  description={Normativa específica que detalla el funcionamiento interno de las Asambleas Generales de Estudiantes (AGE), incluyendo debates, uso de la palabra, mociones y disciplina.}
}

\newglossaryentry{reglamento_electoral}
{
  name={Reglamento Electoral},
  description={Normativa específica que establece los procedimientos para la organización, ejecución y control de los procesos electorales del CEFIS.}
}

\newglossaryentry{comisiones_trabajo}
{
  name={Comisiones de Trabajo},
  description={Órganos de apoyo, de carácter temporal o permanente, creadas por la AGE o la JD, destinadas a estudiar, proponer o ejecutar tareas específicas que contribuyan a los fines del CEFIS.}
}

\newglossaryentry{comision_liquidadora}
{
  name={Comisión Liquidadora},
  description={Órgano designado por la AGE en caso de disolución del CEFIS, encargado de realizar el inventario, balance final, pago de deudas y distribución del patrimonio remanente.}
}

\newglossaryentry{falta_leve}
{
  name={falta leve},
  plural={faltas leves},
  description={Incumplimiento menor de deberes o normativas internas que no causan perjuicio significativo al CEFIS, sus fines o sus agremiados/as.}
}

\newglossaryentry{falta_grave}
{
  name={falta grave},
  plural={faltas graves},
  description={Acción u omisión que atenta seriamente contra los principios, fines o patrimonio del CEFIS, los derechos de otros/as agremiados/as, o el normal funcionamiento de sus órganos.}
}

\newglossaryentry{falta_muy_grave}
{
  name={falta muy grave},
  plural={faltas muy graves},
  description={Falta grave que, por su naturaleza, intencionalidad, daño causado o reincidencia, reviste una especial trascendencia negativa para el CEFIS o sus agremiados/as.}
}


\usepackage{estatuto}

% Metadata
\newcommand{\DocNom}{ESTATUTO DEL\\CENTRO DE ESTUDIANTES DE FÍSICA}
\newcommand{\DocCorto}{Estatuto del CEFIS}
\newcommand{\DocFecha}{\today}
\newcommand{\DocAutor}{Comité Estatutario de la Escuela Profesional de Física}
\newcommand{\DocUniversidad}{Universidad Nacional Mayor de San Marcos}
\begin{document}

% Cover page
\begin{titlepage}
    \begin{center}
        \null%
        \vfill

        {\sffamily\bfseries\fontsize{16}{19}\selectfont \DocNom\par}

        \vspace{0.5cm}

        {\sffamily\Large \DocAutor\par}
        \vspace{0.25cm}
        {\sffamily\small \DocUniversidad\par}

        \vfill
        {\sffamily\normalsize Última modificación: \DocFecha\par}
    \end{center}
\end{titlepage}

% Table of contents
\pagenumbering{roman}
\tableofcontents
\clearpage
\pagenumbering{arabic}

\clearpage
\printglossary[type=\acronymtype, title={Abreviaciones}]

% ----- Estatuto -----

\Titulo{Disposiciones generales}

\Articulo{Denominación}
El gremio estudiantil de la \gls{epf} de la Facultad de Ciencias Físicas de la \gls{unmsm} se denomina \gls{cefis}, cuyas siglas son CEFIS.\@{}

\Articulo{Naturaleza legal}
El \gls{cefis}, como gremio estudiantil único de la \gls{epf}, rige su existencia y funcionamiento bajo el amparo del artículo 188 del Estatuto de la \gls{unmsm}, el derecho de asociación estudiantil (Art. 100.6, Ley Universitaria 30220) y el derecho de asociación (Art. 2, inciso 13, Constitución Política del Perú).

\Articulo{Duración}
La duración del \gls{cefis} es indefinida e inicia sus actividades luego de la aprobación del presente \gls{estatuto} y la elección de su primera \gls{jd}.

\Articulo{Domicilio}
La sede y domicilio legal del \gls{cefis} se ubica en los espacios asignados dentro de la Facultad de Ciencias Físicas, Ciudad Universitaria de la \gls{unmsm}, Av. Venezuela cdra. 34, Cercado de Lima, Lima.

\Articulo[art:principios-cefis]{Principios}
El \gls{cefis} se rige por los siguientes principios:
\begin{artitems}[nosep]
    \artitem{Respeto irrestricto a todos los estudiantes de la \gls{epf}, con igualdad y sin discriminación.}%
    \artitem{Prevalencia del interés estudiantil y los derechos reconocidos en la Constitución, Ley Universitaria, Estatuto \gls{unmsm} y este \gls{estatuto}.}
    \artitem{Autonomía política y administrativa respecto a las autoridades universitarias y otros entes.}
    \artitem{Actitud crítica y racional frente a las propuestas y decisiones de las autoridades pertenecientes a la comunidad universitaria.}
    \artitem{Ética, transparencia y rendición de cuentas en todas sus acciones y decisiones. Acceso a la información pública del gremio garantizado.}
    \artitem{Cultura democrática y de solidaridad con la comunidad sanmarquina.}
    \artitem{Reconocimiento y fomento de la participación voluntaria y el compromiso de los/las \glspl{agremiado} en las actividades y la gestión del \gls{cefis}.}
\end{artitems}

\Articulo[art:fines-obj-cefis]{Fines y objetivos}
Los fines y objetivos del \gls{cefis} son:
\begin{artitems}
    \artitem{Representar y defender los derechos e intereses individuales y colectivos de los estudiantes de la \gls{epf}.}
    \artitem[item:fines-obj-cefis:desarrollo-academico]{Promover y contribuir al desarrollo académico-intelectual, cultural y recreativo dentro de la formación de los estudiantes.}
    \artitem{Fiscalizar la gestión de las autoridades competentes para asegurar las condiciones adecuadas para las actividades académicas y estudiantiles.}
    \artitem{Fomentar la mejora constante y progresiva de la calidad educativa y de la plana docente.}
    \artitem{Promover la participación informada y activa del estudiantado en la vida universitaria y en las decisiones que le conciernen.}
    \artitem{Forjar y mantener una actitud crítica y una conducta consciente y comprometida con la problemática universitaria y educativa nacional.}
    \artitem{Establecer y mantener vínculos de cooperación con otros gremios estudiantiles y organizaciones afines, dentro y fuera de la \gls{unmsm}, en concordancia con los principios y fines del \gls{cefis}.}
\end{artitems}

\Titulo{Del patrimonio y recursos}

\Articulo[art:patrimonio-cefis]{Patrimonio del CEFIS}
El patrimonio del \gls{cefis} comprende todos los bienes muebles e inmuebles, recursos materiales e inmateriales, y rentas derivadas, adquiridos legítimamente (compra, donación, etc.). Su administración corresponde a la \gls{jd} vigente, orientada por los fines y principios del \gls{estatuto} y sujeta a rendición de cuentas.

\Articulo[art:inventario-bienes]{Inventario de bienes}
\begin{artitems}
    \artitem{Todo el patrimonio del \gls{cefis} se registra detalladamente en un inventario de bienes, el cual incluye una descripción precisa de cada bien, su estado de conservación y la fecha de adquisición o recepción.}
    \artitem{El registro es inmediato tras la adquisición o recepción. El inventario se actualiza y verifica físicamente al menos semestralmente.}
    \artitem{La Secretaría de Actas y Economía es responsable de mantener actualizado y accesible el inventario como parte del \gls{registro_central}.}
\end{artitems}

\Articulo[art:cefis-ambientes]{Uso de ambientes}
El \gls{cefis} tiene derecho al uso y administración de los espacios físicos cedidos por la Facultad para sus actividades. El uso se rige por este \gls{estatuto} y reglamentos específicos aprobados por la \gls{jd} para cada espacio, los cuales deben ser publicados y respetados. Para espacios con gestión particular, como bibliotecas estudiantiles o ambientes de representantes estudiantiles, los reglamentos de uso se elaboran en coordinación con sus responsables designados o \glspl{comisiones_trabajo} gestoras, si las hubiere, y son aprobados por la \gls{jd}. La \gls{jd} es responsable de velar por el correcto uso general de los ambientes.

\Articulo[art:recursos-financieros]{Recursos financieros}
La administración y control de los recursos financieros del \gls{cefis} se lleva a cabo conforme a las siguientes disposiciones:
\begin{artitems}
    \artitem[item:recursos-financieros:registro-movimientos]{Todo ingreso y egreso se registra en un \gls{libro_contable} (físico o digital) dentro de los tres (3) días hábiles de ocurrida la transacción. El registro incluye fecha y hora de la transacción, concepto, monto y comprobante válido (factura, boleta, recibo, declaración jurada simple para gastos menores justificados). La Secretaría de Actas y Economía es responsable de este registro, que forma parte del \gls{registro_central}.}
    \artitem{La Secretaría de Actas y Economía gestiona las cuentas bancarias o plataformas de pago (como Yape u otras aprobadas por la \gls{jd}) a nombre del \gls{cefis}. Solicita y archiva en el \gls{registro_central} los estados de cuenta o reportes mensuales.}
    \artitem[item:recursos-financieros:balances]{Se presenta un Balance General semestral a la \gls{jr} y anual a la \gls{age}. Toda la documentación financiera (\gls{libro_contable}, balances, reportes de cuentas, comprobantes) es parte del \gls{registro_central} y accesible para consulta de cualquier \gls{agremiado_habilitado}, siguiendo el procedimiento establecido por la \gls{jd} y facilitado por la Secretaría de Actas y Economía.}
    \artitem[item:recursos-financieros:autorizacion-egresos]{Los egresos deben estar alineados al presupuesto aprobado por la \gls{age} y ser autorizados formalmente por la \gls{jd} (registrado en acta de \gls{jd}), con firma conjunta de la persona titular de la Secretaría General y de la Secretaría de Actas y Economía para montos que superen el umbral definido en el reglamento interno de la \gls{jd} o, en su defecto, por acuerdo específico y registrado de la \gls{jd}.}
\end{artitems}

\Articulo[art:canales-comunicacion]{Canales oficiales de comunicación}
El \gls{cefis} tiene bajo su titularidad todos los canales de comunicación que requiera pertinentes para el cumplimiento de sus fines. Los principales incluyen:
\begin{artitems}
    \artitem{Un correo electrónico oficial para la recepción y envío de comunicaciones formales.}
    \artitem{Los perfiles oficiales en plataformas de redes sociales para la difusión de actividades, eventos y anuncios importantes.}
    \artitem{Un tablón de anuncios físico en un lugar visible y accesible en la Facultad.}
    \artitem{Publicación de actas de \gls{age} y comunicados oficiales.}
\end{artitems}
La Secretaría de Prensa y Difusión es responsable de la gestión, mantenimiento y uso adecuado de estos canales, asegurando la difusión oportuna y veraz de la información oficial. Las comunicaciones oficiales deben archivarse en el \gls{registro_central}.

\Titulo{De los agremiados del CEFIS}

\Articulo[art:definicion-agremiado]{Condición de agremiado/a}
Son \glspl{agremiado} del \gls{cefis} todos los estudiantes de pregrado con matrícula vigente en el semestre académico regular en la \gls{epf}; esto incluye a quienes cuentan con reserva de matrícula válida o se encuentran en proceso de reactualización de matrícula. Se considera \gls{agremiado_habilitado} a quien cumple esta condición y no tiene suspendidos sus derechos conforme a este \gls{estatuto}.

\Articulo{Pérdida de la condición de agremiado/a}
Se pierde la condición de \gls{agremiado} por:
\begin{artitems}
    \artitem[item:perdida-condicion-agremiado:egreso]{Concluir los estudios de pregrado en la \gls{epf} (egreso).}
    \artitem{Renuncia voluntaria, comunicada por escrito (físico o correo electrónico oficial) a la \gls{jd}, la cual registra la renuncia.}
    \artitem{Fallecimiento.}
    \artitem{Exclusión por \gls{falta_muy_grave}, acordada por la \gls{age} conforme al procedimiento establecido en el \aref{art:age-procedimiento-sancionador} y la sanción contemplada en el \aref{item:age-sanciones-aplicables:exclusion-agremiado}.}
    \artitem[item:perdida-condicion-agremiado:definitiva]{Pérdida definitiva de la condición de estudiante de la \gls{epf} (traslado, expulsión universitaria firme).}
\end{artitems}

\Articulo[art:deberes-agremiados]{Deberes de los agremiados/as}
Son deberes de los \glspl{agremiado}:
\begin{artitems}
    \artitem{Conocer, respetar y cumplir el presente \gls{estatuto}, los reglamentos internos y los acuerdos válidamente adoptados por los órganos del \gls{cefis}.}
    \artitem{Respetar y defender los derechos de todos los demás \glspl{agremiado} y miembros de la comunidad universitaria.}
    \artitem{Participar activamente en las asambleas y actividades convocadas por el \gls{cefis}.}
    \artitem{Actuar con ética y probidad en el ámbito universitario y en representación del \gls{cefis} si fuera el caso.}
    \artitem{Contribuir al logro de los fines del \gls{cefis}.}
    \artitem{Utilizar responsablemente los bienes, servicios y canales de comunicación del \gls{cefis}.}
    \artitem{Informar a la instancia correspondiente (\gls{jr} o \gls{jd}) sobre actos que presuntamente atenten contra este \gls{estatuto}, los principios del \gls{cefis} o los derechos de los \glspl{agremiado}.}
\end{artitems}

\Articulo{Derechos de los agremiados/as}
Son derechos de los \glspl{agremiado_habilitado}:
\begin{artitems}
    \artitem{Participar con voz y voto en las Asambleas Generales y de Base.}
    \artitem{Elegir y ser elegido/a para los órganos del \gls{cefis}, cumpliendo los requisitos y sin discriminación.}%
    \artitem{Expresar libremente sus ideas y opiniones en el marco del respeto mutuo y el \gls{estatuto}.}
    \artitem{Solicitar y recibir información sobre la gestión, finanzas y actuación de los órganos del \gls{cefis}, accediendo al \gls{registro_central} conforme a las normas, y fiscalizar dicha gestión.}
    \artitem{Utilizar los servicios e instalaciones del \gls{cefis} conforme a la normativa interna.}
    \artitem{Presentar propuestas e iniciativas a los órganos del \gls{cefis}.}
    \artitem{Exigir el cumplimiento de sus derechos estudiantiles a través del \gls{cefis}.}
    \artitem{A la protección de sus datos personales conforme a ley (Ley N 29733), usados solo para fines estatutarios.}
    \artitem[item:derechos-agremiados:proteccion-denuncias]{A recibir protección contra represalias por denunciar irregularidades de buena fe de conformidad con el \aref{item:jr-atribuciones:velar-proteccion-represalias} y el \aref{item:denuncia-mecanismo:proteccion-buena-fe}.}
\end{artitems}

\Titulo{De la administración y gobierno del CEFIS}

\Articulo[art:organos-gobierno]{Órganos de gobierno}
Constituyen órganos del \gls{cefis}:
\begin{artitems}
    \artitem{La \glsentryfull{age};}
    \artitem{La \glsentryfull{jr};}
    \artitem{La \glsentryfull{jd};}
    \artitem{La \glsentryfull{ab}; y}
    \artitem{El \glsentryfull{ce}.}
\end{artitems}

\Articulo[art:registro-central]{El Registro Central del CEFIS}
\begin{artitems}
    \artitem{Se establece el \gls{registro_central} como el archivo organizado (físico y/o digital) de toda la documentación oficial del gremio.}
    \artitem{La Secretaría de Actas y Economía es responsable de su administración, organización, custodia, actualización y de facilitar el acceso controlado según las normas. La Secretaría General supervisa su correcto funcionamiento.}
    \artitem{El \gls{registro_central} contiene, como mínimo:}
    \begin{enumerate}
        \subartitem{El presente \gls{estatuto} y sus modificaciones.}
        \subartitem{Los Reglamentos internos aprobados (Electoral, de Asambleas, de uso de ambientes, etc.).}
        \subartitem{Las Actas de las \glspl{age}.}
        \subartitem{Las Actas de las sesiones de la \gls{jd}.}
        \subartitem{Las Actas de las sesiones de la \gls{jr}.}
        \subartitem{Las Resoluciones y Actas del \gls{ce}.}
        \subartitem{El Padrón de \glspl{agremiado_habilitado} (actualizado).}
        \subartitem{El inventario de bienes actualizado (\aref{art:inventario-bienes}).}
        \subartitem{La documentación financiera completa (\gls{libro_contable}, balances, reportes de cuentas, comprobantes) (\aref{art:recursos-financieros}).}
        \subartitem{Los Planes Anuales de Trabajo y Presupuestos aprobados.}
        \subartitem{Los informes de gestión (\gls{jd}, \gls{jr}, Comisiones).}
        \subartitem{Comunicaciones oficiales emitidas y recibidas.}
        \subartitem{Decisiones documentadas de la \gls{jd} (incluyendo urgentes y sus ratificaciones).}
        \subartitem{Registro de solicitudes de convocatoria a \gls{age} y sus respuestas.}
        \subartitem{Registro de denuncias, informes de investigación preliminar (\gls{jr}) y decisiones finales de \gls{age} en materia disciplinaria (conforme \aref{art:sanciones-registro-confidencial} sobre confidencialidad del registro de sanciones, y la debida reserva en investigaciones según \aref{subitem:investigacion-preliminar-proceso:alcance}).}
        \subartitem{Registro de apelaciones, informes de revisión y decisiones finales (conforme \aref{art:elecciones-impugnaciones-apelaciones}, \aref{art:apelaciones-no-disciplinarias-generales}).}
        \subartitem{Registro de las decisiones urgentes adoptadas por la \gls{jd} conforme al \aref{item:jd-atribuciones-generales:decisiones-urgentes}, incluyendo su justificación, informe a la \gls{jr} y acta de ratificación por la \gls{age}, si corresponde.}%
        \subartitem{Registro de las solicitudes y decisiones de anulación por razones éticas adoptadas por la \gls{age} conforme al \aref{art:anulacion-razones-eticas}, incluyendo su fundamentación.}
        \subartitem{Actas de transferencia de cargo (\aref{art:jd-informe-transferencia}).}
        \subartitem{Convenios o acuerdos con otras organizaciones.}
    \end{enumerate}
    \artitem{La información pública del Registro (estatutos, reglamentos, actas de \gls{age}, balances aprobados, planes de trabajo, informes de gestión públicos) es de libre acceso para los \glspl{agremiado}. El acceso a información sensible (datos personales, detalles financieros específicos, procesos disciplinarios confidenciales) es restringido según normativa y gestionado por la Secretaría de Actas y Economía.}
\end{artitems}

\Capitulo{De la Asamblea General de Estudiantes (AGE)}

\Articulo{Definición y composición}
La \gls{age} es la máxima instancia deliberativa y decisoria del \gls{cefis}. Está integrada por la totalidad de los \glspl{agremiado_habilitado} conforme al \aref{art:definicion-agremiado}.

\Articulo{Tipos de Asamblea}
Las \glspl{age} se clasifican en:
\begin{artitems}
    \artitem{\gls{ageo}: Se realizan obligatoriamente dos (2) veces por semestre académico regular.}
    \artitem{\gls{agee}: Se convocan cuando sea necesario, conforme a este \gls{estatuto}.}
\end{artitems}

\Articulo[art:age-convocatoria]{Convocatoria}
La convocatoria a \gls{age} es responsabilidad primaria de la Secretaría General de la \gls{jd}, y procede en los siguientes casos:
\begin{artitems}
    \artitem{Acuerdo registrado en acta de la \gls{jd}.}
    \artitem{Acuerdo registrado en acta de la \gls{jr}.}
    \artitem{Solicitud formal y registrada del \gls{ce}, para asuntos de su competencia exclusiva.}
    \artitem{Solicitud formal (escrita o vía correo oficial del \gls{cefis}) de al menos el cinco por ciento (5\%) de los \glspl{agremiado_habilitado}, dirigida a la \gls{jd}, especificando agenda. La \gls{jd} registra la recepción de la solicitud.}
\end{artitems}

\Articulo{Plazos y procedimiento de convocatoria}
\begin{artitems}
    \artitem[item:age-plazos-proc-convocatoria:emision]{Recibida una solicitud válida (conforme \aref{art:age-convocatoria}, numerales 2, 3 o 4), la \gls{jd} tiene un plazo máximo de cinco (5) días hábiles para emitir la convocatoria o denegarla mediante decisión motivada y registrada en el \gls{registro_central}. La denegación se comunica al solicitante y a la \gls{jr} en el mismo plazo.}
    \artitem[item:age-plazos-proc-convocatoria:anticipacion]{La convocatoria se publica con la siguiente anticipación mínima:}
    \begin{enumerate}
        \subartitem{\glsentryfull{ageo}: Diez (10) días calendario.}
        \subartitem{\glsentryfull{agee}: Tres (3) días calendario.}
    \end{enumerate}
    \artitem{La convocatoria indica fecha, hora, lugar/plataforma, modalidad (presencial, virtual, híbrida), agenda detallada, y si aplica la segunda llamada con quórum reducido. Se difunde por todos los canales oficiales del \gls{cefis} (\aref{art:canales-comunicacion}).}
\end{artitems}

\Articulo[art:age-convocatoria-subsidiaria]{Convocatoria subsidiaria}
Si la persona titular de la Secretaría General no convoca en el plazo (\aref{item:age-plazos-proc-convocatoria:emision}) o la deniega sin justificación válida (evaluada por la \gls{jr} en tres (3) días hábiles, cuya decisión se registra), la \gls{jr} puede convocar directamente la \gls{age}, cumpliendo los plazos de difusión (\aref{item:age-plazos-proc-convocatoria:anticipacion}). La convocatoria y la decisión de la \gls{jr} se registran en el \gls{registro_central}.

\Articulo{Convocatoria de urgencia}
En casos excepcionales de extrema urgencia debidamente justificada por la \gls{jd} o la \gls{jr}, que afecten gravemente los intereses colectivos de los estudiantes, la \gls{agee} podrá convocarse con anticipación menor a la establecida, incluso para el mismo día. La justificación y convocatoria deben documentarse, registrarse y difundirse de inmediato. La validez de la urgencia debe ser ratificada por la propia \gls{age} al inicio de la sesión, constando en acta.

\Articulo[art:age-quorum-1ra-llamada]{Quórum de instalación (Primera convocatoria --- primera llamada)}
Se requiere la presencia verificada (física o virtual) y registrada en acta de:
\begin{artitems}
    \artitem{La persona titular de la Secretaría General y de la Secretaría de Actas y Economía de la \gls{jd} (o sus reemplazos válidos, \aref{art:jd-cooperacion-suplencia}).}
    \artitem{Personas titulares de \glspl{delegatura} o \glspl{subdelegatura} que representen a la mayoría simple (mitad más uno) de las \glspl{base_activa} (definidas en \aref{art:jr-definicion-composicion}).}
    \artitem{Un mínimo de quince (15) \glspl{agremiado_habilitado}, incluyendo a quienes se mencionan en los numerales anteriores.}
\end{artitems}

\Articulo[art:age-quorum-2da-llamada]{Quórum de instalación (Primera convocatoria --- segunda llamada)}
Treinta (30) minutos después de la hora fijada, si no se alcanza el quórum inicial, la \gls{age} se instala válidamente con la presencia verificada y registrada de:
\begin{artitems}
    \artitem{La persona titular de la Secretaría General y de la Secretaría de Actas y Economía (o reemplazos).}
    \artitem{Personas titulares de \glspl{delegatura} o \glspl{subdelegatura} que representen al menos a un tercio (1/3) de las \glspl{base_activa}.}
    \artitem{Un mínimo de doce (12) \glspl{agremiado_habilitado} presentes en total, incluyendo a quienes se mencionan en los numerales anteriores.}
\end{artitems}
La posibilidad de segunda llamada debe constar en la convocatoria.

\Articulo[art:age-quorum-2da-convocatoria]{Quórum de instalación (Segunda convocatoria --- reprogramada)}
Si no se alcanza quórum en segunda llamada, la \gls{jd} (o la \gls{jr} si convocó subsidiariamente) debe convocar a una nueva \gls{age} (segunda convocatoria) para la misma agenda, entre siete (7) y quince (15) días calendario después. Esta segunda convocatoria se instala válidamente con la presencia verificada y registrada de:
\begin{artitems}
    \artitem{La persona titular de la Secretaría General y de la Secretaría de Actas y Economía (o reemplazos).}
    \artitem{Cualquier número de titulares de \glspl{delegatura} o \glspl{subdelegatura} de \glspl{base_activa} presentes.}
    \artitem{Un mínimo de diez (10) \glspl{agremiado_habilitado} presentes en total, incluyendo a quienes se mencionan en los numerales anteriores.}
\end{artitems}
La convocatoria debe indicar expresamente que se trata de una segunda convocatoria y que se instalará con el quórum reducido.

\Articulo{Dirección de la Asamblea}
Preside la \gls{age} la persona titular de la Secretaría General de la \gls{jd}. En su ausencia, la persona titular de la Secretaría Académica. En ausencia de ambas, la \gls{age} designará a un miembro presente de la \gls{jd} o la \gls{jr} para dirigir los debates. La persona titular de la Secretaría de Actas y Economía actúa como secretaría de la Asamblea; en su ausencia, la \gls{age} designa una secretaría ad hoc entre los \glspl{agremiado} presentes. Estas designaciones constan en acta.

\Articulo[art:age-acuerdos-mayorias]{Adopción de acuerdos y mayorías}
\begin{artitems}
    \artitem{Los acuerdos se adoptan por regla general con el voto favorable de la mitad más uno de los \glspl{agremiado_habilitado} presentes al momento de la votación. El número de votantes y el resultado (a favor, en contra, abstenciones) se registra en acta.}
    \artitem{Se requiere el voto favorable de dos tercios de los \glspl{agremiado_habilitado} presentes para:}
    \begin{enumerate}
        \subartitem[subitem:age-acuerdos-mayorias:modificar-estatuto-2tercios]{Modificar el presente \gls{estatuto} (\aref{item:agee-atribuciones:modificar-estatuto}, \aref{item:estatuto-aprobacion-modificacion:mayoria-2tercios}).}
        \subartitem[subitem:age-acuerdos-mayorias:disolucion-cefis-2tercios]{Acordar la disolución del \gls{cefis} (\aref{item:agee-atribuciones:disolver-cefis}, \aref{item:disolucion-procedimiento:voto-2tercios}).}
        \subartitem[subitem:age-acuerdos-mayorias:remocion-jd-2tercios]{Acordar la remoción de miembros de la \gls{jd} por \gls{falta_grave} (\aref{item:agee-atribuciones:evaluar-remover-jd}, \aref{item:age-sanciones-aplicables:remocion-cargo}).}%
        \subartitem[subitem:age-acuerdos-mayorias:exclusion-agremiado-2tercios]{Acordar la exclusión de un/a \gls{agremiado} por \gls{falta_muy_grave} (\aref{item:agee-atribuciones:resolver-faltas-graves}, \aref{item:age-sanciones-aplicables:exclusion-agremiado}).}
        \subartitem[subitem:age-acuerdos-mayorias:anulacion-etica-2tercios]{Anular una decisión por razones éticas (\aref{item:anulacion-razones-eticas:procedimiento-agee}).}
    \end{enumerate}
    \artitem{Las votaciones son públicas y nominales, a mano alzada (presencial) o mediante herramientas electrónicas verificables (virtual/híbrida) que permitan el registro individual del voto, salvo que la Asamblea acuerde por mayoría simple el voto secreto para casos específicos (elecciones internas, asuntos personales sensibles), garantizando siempre la verificación del quórum y el resultado.}
\end{artitems}

\Articulo{Actas, registro y publicación}
\begin{artitems}
    \artitem{La persona titular de la Secretaría de Actas y Economía (o la secretaría ad hoc) es responsable de redactar el acta de cada \gls{age}. El acta consigna lugar/plataforma, fecha, hora de inicio/fin, lista de asistentes verificados, agenda tratada, resumen de debates principales, acuerdos adoptados textualmente y resultados de las votaciones (incluyendo número de votos a favor, en contra y abstenciones).}
    \artitem{El acta es firmada por quien presidió y quien actuó en la secretaría.}
    \artitem[item:age-actas-registro:publicacion-archivo]{El acta es un documento oficial, se publica en los canales oficiales del \gls{cefis} y se archiva en el \gls{registro_central} en un plazo máximo de siete (7) días hábiles tras la sesión. Las actas son accesibles para consulta de todos los \glspl{agremiado}.}
\end{artitems}

\Articulo{Reglamento de Asambleas}
El funcionamiento detallado de las \glspl{age} (debate, uso de la palabra, mociones, disciplina, procedimientos específicos por modalidad, registro de intervenciones) se rige por un \gls{reglamento_asambleas}. Este será propuesto por la \gls{jd} o la \gls{jr} y aprobado/modificado por \gls{age} por mayoría simple. Este reglamento debe ser revisado formalmente al menos cada dos (2) años (\aref{item:ageo-atribuciones:actualizar-criterios-bienal}).

\Articulo{Obligatoriedad y ejecución de acuerdos}
\begin{artitems}
    \artitem{Los acuerdos válidamente adoptados en \gls{age} y registrados en acta son vinculantes para todos los \glspl{agremiado} y órganos del \gls{cefis}.}
    \artitem{La \gls{jd} es la principal responsable de ejecutar los acuerdos, debiendo documentar su avance en el \gls{registro_central}. La \gls{jr} supervisa activamente su cumplimiento.}
\end{artitems}

\Articulo[art:ageo-atribuciones]{Atribuciones de la Asamblea General Ordinaria (AGE-O)}
Son atribuciones principales de la \gls{ageo}:
\begin{artitems}
    \artitem{Elegir a las personas integrantes de la \gls{jd} y del \gls{ce}.}
    \artitem{Aprobar el Plan Anual de Trabajo y el Presupuesto presentados por la \gls{jd} entrante.}
    \artitem{Evaluar y aprobar el informe anual de gestión y el balance económico-patrimonial presentado por la \gls{jd} saliente, previo informe de revisión emitido y registrado por la \gls{jr}.}
    \artitem{Aprobar la creación o disolución de \glspl{comisiones_trabajo} permanentes, a propuesta de \gls{jd} o \gls{jr}, incluyendo comisiones para la gestión de espacios específicos como la Biblioteca Estudiantil.}
    \artitem{Recibir y debatir informes periódicos de gestión de la \gls{jd} y la \gls{jr}.}
    \artitem{Aprobar o modificar el \gls{reglamento_asambleas} y otros reglamentos internos (excepto el \gls{reglamento_electoral}, cuya aprobación o modificación corresponde a la \gls{agee} conforme al \aref{item:agee-atribuciones:aprobar-reglamento-electoral}).}
    \artitem[item:ageo-atribuciones:actualizar-criterios-bienal]{Revisar y actualizar formalmente, al menos cada dos (2) años, los criterios de elegibilidad para cargos, los reglamentos internos (incluyendo Asambleas y Electoral) y otros procedimientos documentados, a propuesta de la \gls{jd} o la \gls{jr}.}
    \artitem{Tratar otros asuntos de gestión ordinaria y planificación que consten en agenda.}
    \artitem{Interpretar auténticamente el \gls{estatuto} (\aref{item:estatuto-interpretacion:final-age}).}
\end{artitems}

\Articulo[art:agee-atribuciones]{Atribuciones de la Asamblea General Extraordinaria (AGE-E)}
Son atribuciones principales de la \gls{agee}:
\begin{artitems}
    \artitem[item:agee-atribuciones:modificar-estatuto]{Modificar el presente \gls{estatuto} (mayoría 2/3, conforme \aref{subitem:age-acuerdos-mayorias:modificar-estatuto-2tercios}).}
    \artitem[item:agee-atribuciones:aprobar-reglamento-electoral]{Aprobar o modificar el \gls{reglamento_electoral}.}
    \artitem[item:agee-atribuciones:evaluar-remover-jd]{Evaluar la gestión y determinar responsabilidades de la \gls{jd} o sus miembros. Acordar su remoción por causa justificada (mayoría 2/3, conforme \aref{subitem:age-acuerdos-mayorias:remocion-jd-2tercios}), previo informe documentado y recomendación de la \gls{jr} o comisión ad hoc designada por la \gls{age}. Se debe garantizar el debido proceso (derecho a descargo, presentación de pruebas), el cual queda registrado en acta (\aref{art:investigacion-preliminar-proceso}).}
    \artitem{Evaluar la actuación de la \gls{jr} y emitir recomendaciones registradas en acta.}
    \artitem{Ratificar o anular decisiones específicas de la \gls{jd} consideradas contrarias al \gls{estatuto}, sus principios, fines del \gls{cefis} o acuerdos previos de \gls{age}. Esto procede a solicitud fundamentada y registrada de la \gls{jr} o del diez por ciento (10\%) de \glspl{agremiado_habilitado}.}
    \artitem[item:agee-atribuciones:resolver-faltas-graves]{Resolver sobre \glspl{falta_grave} o \glspl{falta_muy_grave} de \glspl{agremiado}, pudiendo aplicar sanciones (incluyendo la exclusión, con mayoría de 2/3 conforme al \aref{subitem:age-acuerdos-mayorias:exclusion-agremiado-2tercios}), siguiendo el procedimiento establecido en el \aref{art:age-procedimiento-sancionador}, garantizando debido proceso registrado en acta y considerando informe previo de la \gls{jr}.}
    \artitem{Adoptar decisiones políticas institucionales urgentes ante crisis o situaciones que afecten gravemente al estudiantado, con justificación de urgencia registrada en acta.}
    \artitem{Acordar medidas de fuerza colectivas en defensa de los derechos estudiantiles, previa evaluación y recomendación fundamentada y registrada por la \gls{jr}.}
    \artitem{Aprobar acuerdos de cooperación o afiliación con otras organizaciones.}
    \artitem[item:agee-atribuciones:disolver-cefis]{Acordar la disolución y liquidación del \gls{cefis} (mayoría 2/3, conforme \aref{subitem:age-acuerdos-mayorias:disolucion-cefis-2tercios}).}
    \artitem{Cubrir vacancias en la \gls{jd} o el \gls{ce}.}
    \artitem[item:agee-atribuciones:resolver-apelacion-ce]{Resolver, en instancia final y definitiva, apelaciones sobre resoluciones del \gls{ce} únicamente por casos de grave irregularidad probada que vicie sustancialmente el proceso (\aref{item:ce-atribuciones:decisiones-efectivas}, \aref{item:elecciones-impugnaciones-apelaciones:agee-final-instancia}). La decisión debe ser motivada y registrada en acta.}
    \artitem{Resolver apelaciones sobre decisiones administrativas de la \gls{jd} que la \gls{jr} eleve (\aref{item:apelaciones-no-disciplinarias-generales:elevacion-age}).}
    \artitem{Resolver solicitudes de anulación de decisiones por razones éticas (\aref{art:anulacion-razones-eticas}).}
    \artitem{Resolver cualquier otro asunto de trascendencia no previsto o no competencia de la \gls{ageo}, siempre que conste en agenda.}
\end{artitems}

\Capitulo{De la Junta de Representantes (JR)}

\Articulo[art:jr-definicion-composicion]{Definición y composición}
La \gls{jr} es el órgano colegiado permanente de coordinación de bases, fiscalización de la \gls{jd} y representación estudiantil articulada. Está integrada por:
\begin{artitems}
    \artitem{Representantes estudiantiles titulares de la \gls{epf} ante el Consejo Universitario, Asamblea Universitaria y Consejo de Facultad.}
    \artitem{Dos (2) representantes estudiantiles titulares ante el Comité de Gestión de la \gls{epf}.}
    \artitem{La persona titular de la \gls{delegatura} de cada \gls{base_activa}.}
    \artitem{La persona titular de la \gls{subdelegatura} de cada \gls{base_activa}.}
\end{artitems}
Todos sus miembros participan con voz y voto. Se considera ``\gls{base_activa}'' a cada uno de los cinco (5) años académicos de ingreso más recientes con estudiantes matriculados en el semestre vigente. La Secretaría de Actas y Economía mantiene un registro actualizado de las \glspl{base_activa} y las personas titulares de sus \glspl{delegatura} y \glspl{subdelegatura} electas.

\Articulo{Mandato y naturaleza}
Los miembros ejercen su cargo en la \gls{jr} mientras ostenten la representación original que les dio acceso. Actúan colegiadamente, como nexo entre bases, \gls{jd} y órganos de gobierno universitario, velando por los intereses generales del \gls{cefis} y los específicos de sus representados.

\Articulo{Coordinación y sesiones}
\begin{artitems}
    \artitem{La \gls{jr} elige entre las personas titulares de \glspl{delegatura}/\glspl{subdelegatura} de base a una persona para la Coordinación de la \gls{jr} por seis (6) meses, renovable consecutivamente una vez. La persona titular de la Coordinación preside sesiones, modera debates, representa a la \gls{jr} y gestiona el registro de sus actividades (actas). La elección se registra en acta de \gls{jr}.}
    \artitem{La \gls{jr} elige entre sus miembros a una persona para la Secretaría de la \gls{jr} por el mismo período que la Coordinación, responsable de llevar el libro de actas (físico/digital) de la \gls{jr}. Las actas se archivan en el \gls{registro_central} y son accesibles.}
    \artitem{Sesiona ordinariamente al menos una (1) vez al mes durante el semestre académico, y extraordinariamente cuando la convoque la persona titular de la Coordinación, la \gls{jd}, o un tercio (1/3) de sus miembros. La convocatoria indica la agenda y se realiza con al menos 48 horas de anticipación para sesiones ordinarias y 24 horas para extraordinarias, salvo urgencia justificada y registrada.}
    \artitem{El quórum para sesionar es la mitad más uno de sus miembros. Los acuerdos se adoptan por mayoría simple de los presentes y se registran en actas.}
    \artitem{La \gls{jd} (al menos la persona titular de la Secretaría General o Académica) debe participar en las sesiones ordinarias de la \gls{jr} con voz pero sin voto, para informar, coordinar y recibir retroalimentación. Su asistencia o inasistencia justificada se registra en acta.}
\end{artitems}

\Articulo{Atribuciones de la Junta de Representantes}
Son atribuciones de la \gls{jr}:
\begin{artitems}
    \artitem{Coordinar acciones y posiciones entre las bases, la \gls{jd} y los representantes ante órganos de gobierno universitario.}
    \artitem{Supervisar y fiscalizar permanentemente la gestión de la \gls{jd}, velando por el cumplimiento del \gls{estatuto}, principios, reglamentos, acuerdos de \gls{age} y el Plan de Trabajo. Puede solicitar informes periódicos de avance.}
    \artitem{Revisar el informe semestral de gestión y balance de la \gls{jd}. Emitir informe de revisión (conformidad u observaciones) registrado para el informe anual que va a la \gls{ageo}.}
    \artitem{Recibir, registrar, canalizar y hacer seguimiento documentado a propuestas, problemáticas y denuncias provenientes de las bases y \glspl{agremiado}.}
    \artitem{Solicitar informes específicos y documentados a la \gls{jd} sobre su gestión, con plazo adecuado para la respuesta, el cual no será menor a tres (3) ni mayor a siete (7) días hábiles, salvo complejidad justificada. La solicitud y la respuesta (o su ausencia) se registran.}
    \artitem{Emitir opiniones, recomendaciones y alertas, debidamente fundamentadas y registradas, a la \gls{jd} y a la \gls{age}.}
    \artitem{Convocar subsidiariamente a \gls{age} (\aref{art:age-convocatoria-subsidiaria}).}
    \artitem{Evaluar y validar o invalidar, mediante acuerdo registrado, la motivación de la \gls{jd} para denegar una convocatoria a \gls{age} solicitada por terceros.}
    \artitem[item:jr-atribuciones:investigacion-preliminar-faltas]{Realizar investigaciones preliminares sobre denuncias de presuntas \glspl{falta_grave}/\glspl{falta_muy_grave} (\aref{art:investigacion-preliminar-proceso}), garantizando el derecho a descargo inicial de la persona investigada. Elevar un informe documentado con conclusiones y recomendación (archivar o iniciar procedimiento) a la \gls{jd} para convocatoria de \gls{agee} si corresponde. Todo el proceso se registra.}
    \artitem{Proponer temas para la agenda de las \glspl{age}.}
    \artitem{Proponer candidatos para las comisiones ad hoc designadas por \gls{age}.}
    \artitem{Evaluar la pertinencia y viabilidad de medidas de fuerza, emitiendo una recomendación fundamentada y registrada a la \gls{agee}.}
    \artitem{Elaborar y proponer a la \gls{age} modificaciones al \gls{estatuto} o reglamentos internos.}
    \artitem{Recomendar a la \gls{age} la revisión periódica (al menos bienal) de criterios, reglamentos y procedimientos.}
    \artitem[item:jr-atribuciones:velar-proteccion-represalias]{Velar por la protección contra represalias de \glspl{agremiado} que participen, denuncien o informen de buena fe sobre irregularidades o problemas. Puede canalizar preocupaciones a instancias pertinentes si fuera necesario, manteniendo la confidencialidad si se requiere y registrando la acción tomada.}
    \artitem{Conducir o supervisar revisiones posincidente (\aref{art:revisiones-posincidente}).}
    \artitem{Revisar apelaciones sobre decisiones administrativas de la \gls{jd} (\aref{item:apelaciones-no-disciplinarias-generales:revision-jr}).}
    \artitem{Las demás que le asigne el presente \gls{estatuto} o la \gls{age}.}
\end{artitems}

\Capitulo{De la Junta Directiva (JD)}

\Articulo{Naturaleza}
La \gls{jd} es el órgano ejecutivo y administrativo del \gls{cefis}. Es responsable de la gestión diaria, la representación institucional y la ejecución de los acuerdos de la \gls{age} y las coordinaciones con la \gls{jr}. Rinde cuentas de su gestión ante la \gls{jr} de forma permanente y la \gls{age} de forma periódica y final.

\Articulo[art:jd-composicion]{Composición}
La \gls{jd} se compone de seis (6) cargos, cuyas personas titulares dirigen las siguientes Secretarías:
\begin{artitems}
    \artitem{Secretaría General;}
    \artitem{Secretaría Académico;}
    \artitem{Secretaría de Actas y Economía;}
    \artitem{Secretaría de Prensa y Difusión;}
    \artitem{Secretaría de Eventos y Logística;}
    \artitem{Secretaría de Bienestar y Género.}
\end{artitems}

\Articulo{Mandato y reelección}
\begin{artitems}
    \artitem{La \gls{jd} es elegida en \gls{ageo} por un período de un (1) año.}
    \artitem{Las personas miembros pueden ser reelegidas consecutivamente por una (1) sola vez, pero para una Secretaría diferente dentro de la \gls{jd}.}
    \artitem{Ningún/a \gls{agremiado} puede formar parte de la \gls{jd} por más de dos (2) períodos en total durante su vida estudiantil de pregrado.}
    \artitem{Si al concluir el período no se ha elegido una nueva \gls{jd} por causas de fuerza mayor debidamente justificadas, cuya validez es determinada por la \gls{jr} mediante acuerdo registrado, la \gls{jd} saliente continúa en funciones con el mandato exclusivo y urgente de colaborar con el \gls{ce} para convocar y realizar el proceso electoral en el plazo más breve posible, no mayor a treinta (30) días calendario, bajo supervisión directa de la \gls{jr}.}
\end{artitems}

\Articulo[art:jd-requisitos]{Requisitos para ser miembro de la JD}
Se requiere, con verificación documental por el \gls{ce}:
\begin{artitems}
    \artitem{Ser \gls{agremiado_habilitado} del \gls{cefis} (\aref{art:definicion-agremiado}).}
    \artitem{Haber aprobado como mínimo treinta y seis (36) créditos del Plan de Estudios de Física.}
    \artitem{No estar incurso en los impedimentos señalados en el \aref{art:jd-impedimentos}.}
    \artitem{Presentar una declaración jurada simple, cuyo formato es proporcionado por el \gls{ce}, de no tener impedimentos y de compromiso con los fines y principios del \gls{cefis}.}
\end{artitems}

\Articulo[art:jd-impedimentos]{Impedimentos para ser miembro de la Junta Directiva}
No pueden ser elegidas ni ejercer cargos en la \gls{jd}:
\begin{artitems}
    \artitem{Quienes tengan matrícula observada por motivos académicos o estén enfrentando un proceso disciplinario formal por \gls{falta_grave} ante instancias universitarias al momento de la inscripción o durante el mandato.}
    \artitem{Quienes ocupen simultáneamente un cargo como miembro titular de la \gls{jr} (\aref{art:jr-definicion-composicion}).}
    \artitem{Quienes hayan sido sancionados/as con remoción de cargo directivo previo en el \gls{cefis} o exclusión como \glspl{agremiado}, por decisión firme de la \gls{age} en los últimos dos (2) años.}
    \artitem{Quienes formen parte del \gls{ce} en el mismo proceso electoral.}
\end{artitems}

\Articulo{Funcionamiento colegiado y subordinación}
La \gls{jd} actúa de forma colegiada y coordinada. Sus decisiones y acciones deben estar alineadas con el \gls{estatuto}, los acuerdos de la \gls{age} y el Plan de Trabajo aprobado. Se subordina a las decisiones de la \gls{age} y está sujeta a la fiscalización de la \gls{jr}. Todas sus decisiones significativas (acuerdos, resoluciones, autorizaciones, contrataciones) deben registrarse en actas de \gls{jd} o en el \gls{registro_central} según corresponda.

\Articulo{Sesiones de la JD}
\begin{artitems}
    \artitem{La \gls{jd} sesiona ordinariamente al menos cada quince (15) días calendario durante el semestre académico, y extraordinariamente cuando lo convoque la persona titular de la Secretaría General o la mayoría simple (mitad más uno) de sus miembros.}
    \artitem{La convocatoria a sesión ordinaria se realiza con una anticipación mínima de 48 horas, y para sesión extraordinaria con 24 horas, salvo urgencia justificada y registrada. La convocatoria incluye la agenda.}
    \artitem{El quórum para sesionar es la mitad más uno de sus miembros titulares (mínimo cuatro). Los acuerdos se adoptan por mayoría simple de los asistentes. En caso de empate, la persona titular de la Secretaría General tiene voto dirimente, el cual se registra expresamente.}
    \artitem{Las sesiones son reservadas para sus miembros, pero podrán invitar a miembros de la \gls{jr} u otros con fines específicos, dejando constancia en acta. La \gls{jd} informa periódicamente sus acuerdos a la \gls{jr} y a los/as \glspl{agremiado} a través de comunicados oficiales. Sus actas son gestionadas conforme al \aref{art:jd-secretaria-actas-economia} y accesibles para consulta de la \gls{jd} y para la \gls{jr}.}
\end{artitems}

\Articulo{Asistencia y abandono de cargo}
\begin{artitems}
    \artitem{La asistencia a las sesiones de la \gls{jd} es obligatoria y se registra en acta.}
    \artitem{La inasistencia debe justificarse por escrito o correo electrónico a la Secretaría General antes de la sesión, salvo imposibilidad manifiesta debidamente acreditada. La justificación se adjunta o menciona en el acta.}
    \artitem[item:jd-asistencia-abandono:configuracion-abandono]{Tres (3) inasistencias injustificadas consecutivas o cinco (5) alternadas injustificadas en un semestre académico, registradas en acta, constituyen abandono de cargo. La Secretaría General emite una advertencia formal registrada (con copia a la \gls{jd} y \gls{jr}) tras la segunda inasistencia consecutiva o cuarta alternada. Declarado el abandono por la \gls{jd} en acta, se procede conforme al \aref{art:jd-causales-vacancia} y \aref{art:jd-reemplazo-vacancia}.}
\end{artitems}

\Articulo[art:jd-causales-vacancia]{Causales de vacancia}
Son causales de vacancia de un cargo en la \gls{jd}, debidamente acreditadas y documentadas en el \gls{registro_central}:
\begin{artitems}
    \artitem{Fallecimiento.}
    \artitem{Incapacidad física/mental permanente sobrevenida, declarada por la \gls{agee} previo informe médico si corresponde y garantizando la confidencialidad.}
    \artitem{Abandono de cargo, declarado por la \gls{jd} en acta (\aref{item:jd-asistencia-abandono:configuracion-abandono}).}
    \artitem{Sentencia judicial firme por delito doloso.}
    \artitem{Pérdida definitiva de la condición de estudiante de pregrado de la \gls{epf} (\aref{item:perdida-condicion-agremiado:egreso}, \aref{item:perdida-condicion-agremiado:definitiva}).}
    \artitem{Renuncia voluntaria formalmente presentada por escrito al Secretario/a General y aceptada por la \gls{jd} en acta.}
    \artitem{Exclusión como \gls{agremiado} del \gls{cefis}, declarada por \gls{agee} con decisión firme.}
    \artitem{Remoción del cargo por \gls{falta_grave} en sus funciones, declarada por \gls{agee} (mayoría 2/3, conforme \aref{subitem:age-acuerdos-mayorias:remocion-jd-2tercios}) con decisión firme registrada en acta.}
    \artitem{Incurrir sobrevenidamente en impedimento (\aref{art:jd-impedimentos}), verificado y declarado por la \gls{jd} en acta, y comunicado a la \gls{jr}.}
\end{artitems}

\Articulo[art:jd-reemplazo-vacancia]{Declaración y reemplazo por vacancia}
\begin{artitems}
    \artitem{Producida una causal, la \gls{jd} declara la vacancia del cargo en un plazo máximo de cinco (5) días hábiles, mediante resolución registrada, comunicando inmediatamente a la \gls{jr} y a los \glspl{agremiado}.}
    \artitem{La \gls{jd} solicita formalmente a la \gls{jr} la convocatoria de una \gls{agee} en un plazo no mayor a quince (15) días hábiles desde la declaración de vacancia, para elegir a la persona reemplazante que completará el período restante.}
    \artitem{Mientras tanto, la Secretaría General encarga temporalmente las funciones al miembro de la \gls{jd} que considere más idóneo, informando formalmente a la \gls{jr}. Este encargo no excede los treinta (30) días calendario.}
\end{artitems}

\Articulo{Renuncia al cargo}
La renuncia se presenta por escrito a la Secretaría General (o a la Secretaría Académica si renuncia la persona titular de la Secretaría General). La \gls{jd} la acepta formalmente en sesión, dejando constancia en acta, y la comunica a la \gls{jr} y a los \glspl{agremiado} en un plazo de tres (3) días hábiles. Si la persona renunciante tiene procesos por \gls{falta_grave} en curso relacionados a su gestión, la aceptación formal puede supeditarse a la conclusión del proceso, sin afectar la efectividad inmediata de la renuncia para ejercer el cargo. Esto consta en el acta de aceptación.

\Articulo{Atribuciones generales de la Junta Directiva}
Son atribuciones y deberes de la \gls{jd}, cuyas acciones y decisiones principales deben ser documentadas y registradas para fines de transparencia y rendición de cuentas:
\begin{artitems}
    \artitem{Dirigir la gestión ejecutiva y administrativa diaria del \gls{cefis}.}
    \artitem{Ejercer la representación oficial e institucional del \gls{cefis} ante autoridades universitarias y entidades externas, documentando las gestiones realizadas.}
    \artitem{Elaborar y proponer a la \gls{ageo} el Plan Anual de Trabajo y el Presupuesto.}
    \artitem{Ejecutar el Plan Anual de Trabajo y el Presupuesto aprobados.}
    \artitem{Cumplir y hacer cumplir el presente \gls{estatuto}, los reglamentos y los acuerdos adoptados por la \gls{age}.}
    \artitem{Presentar informes de gestión y balance económico-patrimonial semestrales a la \gls{jr} y anuales a la \gls{ageo}.}
    \artitem{Observar e informarse de las medidas, actividades y oficios publicados por la Dirección de la \gls{epf} que afecten al estudiantado, y actuar en consecuencia si procede, documentando las acciones.}
    \artitem{Administrar diligentemente el patrimonio y los recursos financieros (\aref{art:patrimonio-cefis} al \aref{art:recursos-financieros}), manteniendo registros contables claros, actualizados y accesibles en el \gls{registro_central}. Aprobar egresos presupuestados, registrando las autorizaciones respectivas.}
    \artitem{Gestionar los canales de comunicación oficiales (\aref{art:canales-comunicacion}), vía Secretaría de Prensa y Difusión.}%
    \artitem{Convocar a \gls{age} según lo estipulado en este \gls{estatuto}.}
    \artitem{Coordinar permanentemente con la \gls{jr} y atender sus solicitudes de información y fiscalización.}%
    \artitem{Organizar y supervisar el funcionamiento de las secretarías y \glspl{comisiones_trabajo} temporales creadas por la \gls{jd}.}
    \artitem{Fomentar la participación estudiantil en las actividades del \gls{cefis} y la vida universitaria.}
    \artitem{Canalizar las demandas y propuestas estudiantiles ante las autoridades competentes, registrando las gestiones realizadas y sus resultados.}
    \artitem[item:jd-atribuciones-generales:decisiones-urgentes]{Tomar decisiones ejecutivas urgentes no previstas, cuando la situación lo amerite y no sea posible convocar a \gls{agee} de inmediato, actuando en resguardo de los intereses del \gls{cefis} y sus \glspl{agremiado}. La persona titular de la Secretaría General es responsable final de activar este mecanismo. Estas decisiones deben ser:}
    \begin{enumerate}
        \subartitem{Documentadas, incluyendo la justificación de la urgencia y la imposibilidad de convocar a \gls{age}.}
        \subartitem{Registradas en el \gls{registro_central}.}
        \subartitem{Informadas en detalle a la \gls{jr} en un plazo máximo de cuarenta y ocho (48) horas.}
        \subartitem{Sometidas a ratificación en la siguiente \gls{age} (ordinaria o extraordinaria), que debe realizarse en un plazo máximo de quince (15) días calendario desde la decisión.}
        \subartitem{La falta de ratificación por la \gls{age} obliga a la \gls{jd} a justificar su actuación ante la \gls{jr} y, si la \gls{age} lo dispone, revertir los efectos de la decisión en la medida de lo posible. La decisión de la \gls{age} se registra en acta.}
    \end{enumerate}
    \artitem{Velar por el correcto uso de los ambientes asignados al \gls{cefis} (\aref{art:cefis-ambientes}), aplicando los reglamentos internos correspondientes.}
    \artitem{Las demás que le asigne el \gls{estatuto} y la \gls{age} que no sean competencia exclusiva de otros órganos.}
\end{artitems}

\Articulo[art:jd-informe-transferencia]{Informe final y transferencia de cargo}
Al finalizar su mandato, la \gls{jd} saliente presenta ante la \gls{ageo} un informe final de gestión detallado y el balance económico-patrimonial del último periodo, previamente revisado por la \gls{jr} (cuyo informe se adjunta). Realiza la transferencia formal y documentada de cargos, bienes, fondos, archivos físicos y digitales del \gls{registro_central}, claves de acceso y toda información necesaria para la continuidad de la gestión a la \gls{jd} entrante. Esto se hace mediante un Acta de Entrega-Recepción firmada por las personas titulares de las Secretarías Generales y de Actas y Economía salientes y entrantes, en un plazo no mayor a siete (7) días hábiles desde la proclamación oficial de la nueva \gls{jd}. Una copia del acta se remite a la \gls{jr} y se archiva en el \gls{registro_central}.

\Capitulo{De las Secretarías de la Junta Directiva}

Las Secretarías son los órganos ejecutores de la \gls{jd}, cada una con un ámbito de acción definido. La persona titular de cada Secretaría es la máxima responsable de su gestión. Cada Secretaría puede contar con un equipo de apoyo conformado por \glspl{agremiado} voluntarios/as, bajo la coordinación y responsabilidad de la persona titular de la Secretaría, conforme al \aref{item:jd-cooperacion-suplencia:equipos-apoyo}.

\Articulo{Secretario/a General}
La persona titular de la Secretaría General es la máxima representante legal y ejecutiva del \gls{cefis}. Preside la \gls{jd}. Es responsable principal de la rendición de cuentas general. Sus deberes y atribuciones específicas son:
\begin{artitems}
    \artitem{Representar oficialmente al \gls{cefis}.}
    \artitem{Convocar y presidir las sesiones de la \gls{jd} y la \gls{age}.}
    \artitem{Dirigir y supervisar la ejecución del Plan de Trabajo, asegurando el registro documentado de actividades y avances.}
    \artitem{Velar por el cumplimiento del \gls{estatuto} y los acuerdos de la \gls{age} por parte de la \gls{jd}.}
    \artitem{Firmar, conjuntamente con la persona titular de la Secretaría de Actas y Economía, la documentación oficial (actas, balances, contratos, convenios) y las autorizaciones de gasto que superen el umbral definido conforme al \aref{item:recursos-financieros:autorizacion-egresos}.}
    \artitem{Coordinar las acciones de las distintas secretarías, promover el trabajo en equipo y la rendición de cuentas específica de cada secretaría, la cual debe quedar documentada.}
    \artitem{Mantener comunicación constante con la \gls{jr} y las autoridades universitarias.}
    \artitem{Recibir, registrar y canalizar la correspondencia oficial dirigida al \gls{cefis}.}
    \artitem{Ejercer el voto dirimente en las sesiones de la \gls{jd}, dejando constancia en acta.}
    \artitem{Delegar funciones específicas en otros miembros de la \gls{jd} cuando sea necesario, con acuerdo registrado de la \gls{jd} y dejando constancia formal escrita de la delegación y su alcance.}
    \artitem{Activar y justificar formalmente el mecanismo de decisión urgente (\aref{item:jd-atribuciones-generales:decisiones-urgentes}) cuando sea estrictamente necesario.}
    \artitem{Representar al \gls{cefis} en procesos administrativos o judiciales, con autorización previa y registrada de la \gls{jd} o \gls{age} según la naturaleza del proceso. Puede otorgar poderes específicos previo acuerdo registrado de la \gls{jd}.}
\end{artitems}

\Articulo{Secretario/a Académico/a}
La persona titular de la Secretaría Académica es responsable de los asuntos relacionados con la formación académica y los derechos estudiantiles en ese ámbito. Reporta directamente a la persona titular de la Secretaría General. Sus funciones son:
\begin{artitems}
    \artitem{Reemplazar a la persona titular de la Secretaría General en caso de ausencia o impedimento temporal, con comunicación registrada a la \gls{jd} y \gls{jr}.}
    \artitem{Identificar, documentar y proponer soluciones a problemáticas académicas (plan de estudios, calidad docente, métodos de evaluación, infraestructura, etc.), a la \gls{jd} y, a través de ella, a las autoridades.}
    \artitem{Coordinar con las autoridades académicas para la mejora continua de los planes de estudio y la calidad educativa, documentando las gestiones.}
    \artitem{Presentar informes periódicos sobre la situación académica de los estudiantes de la \gls{epf} a la \gls{jd}.}
    \artitem{Organizar y promover actividades académicas complementarias (talleres, seminarios, grupos de estudio, programas de mentoría, etc.), llevando un registro de las mismas.}
    \artitem{Gestionar y difundir información sobre becas, pasantías, movilidad estudiantil y oportunidades académicas.}
    \artitem{Coordinar con los representantes estudiantiles ante órganos de gobierno sobre temas académicos.}%
    \artitem{Fomentar la investigación científica y la participación estudiantil en eventos académicos y científicos.}
    \artitem{Canalizar reclamos y consultas académicas individuales o colectivas ante las instancias correspondientes, manteniendo un registro interno y confidencial de los casos individuales gestionados (registrando tipo de reclamo, gestión realizada y resultado, no datos personales sensibles salvo consentimiento explícito para la gestión).}
\end{artitems}

\Articulo[art:jd-secretaria-actas-economia]{Secretario/a de Actas y Economía}
La persona titular de la Secretaría de Actas y Economía es directamente responsable de la gestión documental (\gls{registro_central}), patrimonial y financiera. Reporta directamente a la persona titular de la Secretaría General. Sus funciones son:
\begin{artitems}
    \artitem{Redactar y custodiar las actas de las sesiones de la \gls{jd} y de la \gls{age}, asegurando su correcta elaboración, publicación (\aref{item:age-actas-registro:publicacion-archivo}), archivo en el \gls{registro_central} y accesibilidad.}
    \artitem{Llevar el registro de asistencia a las sesiones de \gls{jd} y \gls{age}.}
    \artitem{Administrar los recursos financieros (\aref{art:recursos-financieros}) y el presupuesto.}
    \artitem{Llevar el \gls{libro_contable} (físico o digital, según se defina) actualizado en el \gls{registro_central} (\aref{item:recursos-financieros:registro-movimientos}).}
    \artitem{Gestionar las cuentas bancarias o plataformas de pago autorizadas del \gls{cefis}, solicitar reportes mensuales y archivarlos en el \gls{registro_central}.}
    \artitem{Elaborar los balances económicos semestrales y anuales, y presentarlos a la \gls{jd}, la \gls{jr} y la \gls{age}.}
    \artitem{Garantizar la transparencia y facilitar el acceso a la información financiera (\aref{item:recursos-financieros:balances}).}
    \artitem{Custodiar y actualizar el Inventario de Bienes (\aref{art:inventario-bienes}), archivado en el \gls{registro_central}.}
    \artitem{Firmar conjuntamente con la persona titular de la Secretaría General la documentación financiera relevante y autorizaciones de gasto.}
    \artitem{Proponer a la \gls{jd} estrategias para la sostenibilidad financiera y la gestión patrimonial.}
    \artitem{Administrar y mantener el \gls{registro_central} (\aref{art:registro-central}), asegurando su organización, integridad, seguridad y accesibilidad controlada.}
\end{artitems}

\Articulo{Secretario/a de Prensa y Difusión}
La persona titular de la Secretaría de Prensa y Difusión es responsable de la comunicación interna y externa del \gls{cefis}. Reporta directamente a la persona titular de la Secretaría General. Sus funciones son:
\begin{artitems}
    \artitem{Administrar y mantener actualizados los canales oficiales de comunicación (\aref{art:canales-comunicacion}), asegurando la coherencia y oportunidad de la información.}
    \artitem{Diseñar y ejecutar la estrategia de comunicación aprobada por la \gls{jd}.}
    \artitem{Redactar y difundir comunicados, convocatorias, resúmenes informativos de actas (en coordinación con la Secretaría de Actas y Economía), notas y material gráfico sobre las actividades y posiciones del \gls{cefis} de forma clara y oportuna.}
    \artitem{Mantener informados a los/as \glspl{agremiado} sobre asuntos de interés de la vida universitaria.}
    \artitem{Gestionar la imagen institucional.}
    \artitem{Mantener un archivo organizado y accesible de todas las comunicaciones emitidas en el \gls{registro_central}.}
    \artitem{Coordinar con las demás secretarías y comisiones del \gls{cefis} para la difusión eficaz de sus actividades.}
\end{artitems}

\Articulo{Secretario/a de Eventos y Logística}
La persona titular de la Secretaría de Eventos y Logística es responsable de la planificación, organización y ejecución de actividades culturales, deportivas, recreativas y de integración. Reporta directamente a la persona titular de la Secretaría General. Sus funciones son:
\begin{artitems}
    \artitem{Proponer y organizar eventos que contribuyan a los fines del \gls{cefis} (\aref{item:fines-obj-cefis:desarrollo-academico}), llevando un registro de los mismos.}
    \artitem{Gestionar la logística necesaria para las actividades del \gls{cefis} (reserva de espacios, solicitud de equipos, compra de materiales, tramitación de permisos), documentando las gestiones.}
    \artitem{Coordinar con otras organizaciones estudiantiles o externas para la realización de eventos conjuntos, formalizando los acuerdos por escrito si implican uso de recursos o compromisos institucionales.}
    \artitem{Fomentar la participación estudiantil en actividades extracurriculares.}
    \artitem{Administrar el uso de locales y materiales del \gls{cefis} destinados a eventos, velando por su buen estado y llevando un registro de préstamos y devoluciones.}
    \artitem{Elaborar cronogramas y presupuestos para los eventos, en coordinación con la Secretaría de Actas y Economía, y rendir cuentas documentadas de los gastos efectuados.}
\end{artitems}

\Articulo{Secretario/a de Bienestar y Género}
La persona titular de la Secretaría de Bienestar y Género es responsable de promover el bienestar estudiantil integral, la equidad y la atención a problemáticas sociales y de género. Reporta directamente a la persona titular de la Secretaría General. Sus funciones son:
\begin{artitems}
    \artitem{Identificar necesidades y problemáticas relacionadas con el bienestar estudiantil (salud física/mental, apoyo socioeconómico, seguridad, inclusión, accesibilidad), documentándolas de forma agregada y anónima para proponer acciones a la \gls{jd}.}
    \artitem{Promover y coordinar campañas de sensibilización y prevención sobre salud integral, derechos humanos, no violencia, equidad de género, diversidad e interculturalidad.}
    \artitem{Servir como punto de contacto inicial confidencial y de orientación para estudiantes que enfrenten situaciones de acoso, discriminación, violencia o vulnerabilidad. Debe canalizar responsablemente los casos a las instancias especializadas pertinentes de la universidad y/o externas, siempre con el consentimiento informado del estudiante. Mantiene un registro interno y estrictamente confidencial del número y tipo de orientaciones brindadas (sin datos personales identificables) para fines estadísticos y de mejora.}
    \artitem{Organizar actividades que promuevan la integración, la solidaridad y un ambiente de respeto mutuo y seguro.}
    \artitem{Coordinar con los servicios de bienestar universitario de la \gls{unmsm} y otras entidades de apoyo para facilitar el acceso estudiantil a dichos servicios, difundiendo información sobre las mismas.}
    \artitem{Proponer a la \gls{jd} y a la \gls{age} políticas y acciones para promover la igualdad de oportunidades, la inclusión y prevención de la discriminación y violencia en el ámbito estudiantil.}
\end{artitems}

\Articulo[art:jd-cooperacion-suplencia]{Cooperación y suplencia}
\begin{artitems}
    \artitem{Todos los miembros de la \gls{jd} tienen el deber de cooperar entre sí y son corresponsables del cumplimiento de los objetivos generales del Plan de Trabajo.}
    \artitem{En caso de ausencia temporal o impedimento de una persona titular de Secretaría (distinta a la General), la persona titular de la Secretaría General encarga temporalmente sus funciones a otro miembro de la \gls{jd}, informando formalmente de ello en la siguiente sesión de \gls{jd} y dejando constancia en acta. Este encargo no genera doble representación ni doble voto.}
    \artitem[item:jd-cooperacion-suplencia:equipos-apoyo]{Cada Secretaría podrá contar con un equipo de apoyo conformado por \glspl{agremiado} voluntarios/as, quienes colaboran bajo la coordinación y responsabilidad de la persona titular de la Secretaría. La conformación de estos equipos se comunica a la \gls{jd} para su conocimiento y registro. La participación voluntaria es un pilar para el funcionamiento del \gls{cefis}.}
    \artitem{La \gls{jd} podrá proponer a la \gls{age} la creación de \glspl{comisiones_trabajo} temporales (\aref{art:comisiones-naturaleza-creacion}).}
\end{artitems}

\Capitulo{De la Asamblea de Base (AB)}

\Articulo{Definición y composición}
La \gls{ab} es la instancia primaria de participación y deliberación por año académico de ingreso (base). Está compuesta por todos los/as \glspl{agremiado_habilitado} pertenecientes a dicha base. Cada base elige una \gls{delegatura} y una \gls{subdelegatura}.

\Articulo{Funciones de la Asamblea de Base y sus Delegaturas}
Las funciones principales de la \gls{ab} son:
\begin{artitems}
    \artitem{Discutir asuntos académicos, administrativos y de bienestar específicos de su base.}
    \artitem{Elegir democráticamente a las personas titulares de la \gls{delegatura} y \gls{subdelegatura} que los representarán ante la \gls{jr}. Comunicar formalmente la elección (mediante un acta simple firmada por los asistentes o un registro verificable) a la \gls{jr} y a la Secretaría de Actas y Economía para su inclusión en el \gls{registro_central}.}
    \artitem{Proponer temas, iniciativas o problemáticas para ser tratados por la \gls{jr} o la \gls{jd}, a través de sus \glspl{delegatura}.}
    \artitem{Canalizar inquietudes y reclamos colectivos de la base a través de sus \glspl{delegatura}.}
    \artitem{Informarse sobre las actividades y decisiones del \gls{cefis} a través de sus \glspl{delegatura} y los canales de comunicación oficiales.}
    \artitem{Promover la integración y participación activa de la base en las actividades y asambleas del \gls{cefis}.}
\end{artitems}

Las funciones específicas de la persona titular de la \gls{delegatura} de Base, con apoyo de la \gls{subdelegatura}, son:
\begin{artitems}
    \artitem{Convocar y presidir las \glspl{ab}.}
    \artitem{Mantener un canal de comunicación activo y regular con los miembros de su base, informándoles sobre las gestiones realizadas y las decisiones de los órganos del \gls{cefis}.}
    \artitem{Representar los intereses y propuestas de su base ante la \gls{jr}.}
    \artitem{Llevar un registro simple de los acuerdos principales de la \gls{ab} para comunicación interna y para informar a la \gls{jr}.}
\end{artitems}

\Articulo{Convocatoria y sesiones de la Asamblea de Base}
\begin{artitems}
    \artitem{La \gls{ab} es convocada por la persona titular de la \gls{delegatura} de Base, o en su ausencia, por la persona titular de la \gls{subdelegatura}.}
    \artitem{También puede ser convocada a solicitud escrita (física o por medio electrónico verificable) de al menos el diez por ciento (10\%) de los \glspl{agremiado_habilitado} de la base, dirigida a la \gls{delegatura}.}
    \artitem{La convocatoria se realiza con una anticipación razonable (mínimo 24 horas) a través de los grupos de informes de la base.}
    \artitem{La \gls{ab} se reúne ordinariamente al menos una vez por semestre y extraordinariamente según sea necesario.}
    \artitem{Las sesiones son dirigidas por la persona titular de la \gls{delegatura} o, en su ausencia, por la \gls{subdelegatura}. Los acuerdos se adoptan por mayoría simple de los asistentes. La \gls{delegatura} es responsable de llevar un registro simple de los acuerdos principales.}
\end{artitems}

\Articulo{Coordinación y responsabilidad de las Delegaturas}
La persona titular de la \gls{delegatura} y la persona titular de la \gls{subdelegatura} son el nexo oficial entre la \gls{ab} y la \gls{jr}. Son responsables de:
\begin{artitems}
    \artitem{Transmitir fielmente las decisiones, propuestas y preocupaciones de la base a la \gls{jr}.}
    \artitem{Informar diligentemente a su base sobre las gestiones realizadas, los acuerdos de la \gls{jr}, la \gls{jd} y la \gls{age} que sean de interés para la base.}
    \artitem{Fomentar la participación informada de su base en las actividades del \gls{cefis}.}
\end{artitems}

\Capitulo{Del Comité Electoral (CE)}

\Articulo{Naturaleza y autonomía}
El \gls{ce} es el órgano autónomo y temporal encargado de organizar, conducir y supervisar el proceso electoral para la renovación de la \gls{jd} y otros cargos de elección que determine la \gls{age}. Goza de plena independencia funcional respecto a la \gls{jd} y la \gls{jr} en el ejercicio de sus atribuciones electorales. Rinde cuentas directamente a la \gls{age}.

\Articulo{Composición y elección}
\begin{artitems}
    \artitem{El \gls{ce} está compuesto por tres (3) miembros titulares, quienes asumen la Presidencia, Secretaría y Vocalía. Se eligen también dos (2) miembros suplentes. Los cargos específicos (Presidencia, Secretaría, Vocalía) son asignados por el propio \gls{ce} en su primera sesión de instalación.}
    \artitem{Son elegidos por la \gls{ageo} convocada para tal fin, por mayoría simple, entre \glspl{agremiado_habilitado} que cumplan los requisitos (\aref{art:ce-requisitos}) y no tengan impedimentos (conforme al \aref{art:ce-impedimentos}). La postulación es individual.}
    \artitem{Su mandato inicia con su juramentación, la cual se registra en el acta de la \gls{age} que los elige, y culmina con la proclamación oficial y registrada de los resultados electorales y la resolución firme de impugnaciones o apelaciones.}
\end{artitems}

\Articulo[art:ce-requisitos]{Requisitos}
Para ser miembro del \gls{ce} se requiere:
\begin{artitems}
    \artitem{Ser \gls{agremiado_habilitado}.}
    \artitem{Declarar disponibilidad de tiempo para las funciones del cargo.}
    \artitem{No postular a ningún cargo en el proceso electoral que organizará.}
    \artitem{Presentar una declaración jurada de no incurrir en los impedimentos del \aref{art:ce-impedimentos} y de compromiso con la imparcialidad y transparencia del proceso electoral.}
\end{artitems}

\Articulo[art:ce-impedimentos]{Impedimentos}
Están impedidos de ser miembros del \gls{ce}:
\begin{artitems}
    \artitem{Los miembros de la \gls{jd} saliente o en funciones.}
    \artitem{Los miembros de la \gls{jr} en funciones.}
    \artitem{Quienes tengan parentesco hasta segundo grado de consanguinidad o primero de afinidad con algún candidato/a inscrita en alguna lista.}
    \artitem{Quienes hayan sido sancionados por \glspl{falta_grave} por la \gls{age} o la \gls{unmsm} en los últimos dos años, con sanción firme.}
    \artitem{Quienes manifiesten apoyo o rechazo a alguna candidatura específica durante el proceso electoral, desde su postulación al \gls{ce}.}
\end{artitems}

\Articulo{Atribuciones del Comité Electoral}
Son atribuciones del \gls{ce}, cuyas decisiones deben ser justificadas, documentadas y registradas en su propio archivo público:
\begin{artitems}
    \artitem{Elaborar y proponer a la \gls{age} para su aprobación el \gls{reglamento_electoral}, o aplicar el vigente, asegurando su conformidad con este \gls{estatuto} y los principios de transparencia, imparcialidad y equidad. La \gls{age} revisa formalmente el \gls{reglamento_electoral} al menos cada dos (2) años (\aref{item:ageo-atribuciones:actualizar-criterios-bienal}, \aref{item:reglamento-electoral:revision-age-bienal}).}
    \artitem{Convocar a elecciones de acuerdo a los plazos establecidos en el \gls{estatuto} y el \gls{reglamento_electoral}.}
    \artitem{Elaborar, depurar y publicar el \gls{padron_electoral} de miembros habilitados para votar, en coordinación con la Secretaría de Actas y Economía para verificar la condición de \gls{agremiado_habilitado}.}
    \artitem{Inscribir las listas de candidatos que cumplan los requisitos establecidos en el \gls{estatuto} y \gls{reglamento_electoral}, verificando la documentación presentada. Publicar las listas inscritas.}
    \artitem{Resolver tachas e impugnaciones contra actos del proceso electoral o resultados, en primera instancia. Sus resoluciones deben ser motivadas y emitidas dentro de los plazos reglamentarios debidamente notificadas y registradas.}
    \artitem{Organizar la jornada electoral, incluyendo la designación y capacitación de miembros de mesa, la instalación de mesas de sufragio (físicas o virtuales asegurando la identidad del votante y el secreto del voto), la supervisión del acto electoral, el escrutinio de votos y la elaboración de las actas correspondientes.}
    \artitem{Garantizar la transparencia, imparcialidad, neutralidad y legalidad del proceso electoral en todas sus etapas, tomando las medidas necesarias para ello.}
    \artitem{Proclamar los resultados oficiales y a los candidatos electos, mediante acta oficial registrada, una vez resueltas todas las impugnaciones en primera instancia o vencido el plazo para presentarlas.}
    \artitem{Solicitar formalmente y por escrito a la \gls{jd} los recursos logísticos y financieros necesarios y razonables para el cumplimiento de sus funciones, con cargo al presupuesto del \gls{cefis}. La \gls{jd} está obligada a proveerlos de manera oportuna y suficiente, dejando constancia documentada de la transferencia de recursos.}
    \artitem{Administrar su propio registro de decisiones, actas, resoluciones y actuaciones, las cuales son públicas y accesibles a todos los estudiantes.}
    \artitem[item:ce-atribuciones:decisiones-efectivas]{Sus decisiones en materia electoral son de cumplimiento obligatorio. Solo pueden ser revisadas en instancia final por la \gls{agee} convocada específicamente para tal fin (\aref{item:agee-atribuciones:resolver-apelacion-ce}), únicamente en casos de grave irregularidad debidamente probada que vicie sustancialmente el proceso. La apelación debe seguir el procedimiento establecido en el \gls{reglamento_electoral}, y la carga de la prueba recae en quien la impugna.}
\end{artitems}

\Titulo{De los procesos electorales}

\Articulo{Principios electorales}
Los procesos electorales del \gls{cefis} se rigen por los principios de democracia interna, transparencia, equidad entre candidaturas, imparcialidad del \gls{ce}, y el derecho al voto universal, libre, secreto, directo y obligatorio de los \glspl{agremiado_habilitado}.

\Articulo{Padrón electoral}
\begin{artitems}
    \artitem{El \gls{padron_electoral} contiene la lista oficial de \glspl{agremiado_habilitado} para votar (\aref{art:definicion-agremiado}).}
    \artitem{El \gls{ce} es el responsable de elaborar, depurar y publicar el \gls{padron_electoral} preliminar, utilizando la información oficial de matrícula proporcionada por la \gls{epf} y verificada por la Secretaría de Actas y Economía.}
    \artitem{El \gls{ce} resuelve las observaciones de forma justificada y documentada, y publicará el \gls{padron_electoral} definitivo antes de la jornada electoral, conforme al plazo fijado en el \gls{reglamento_electoral}. Este Padrón definitivo es archivado en el \gls{registro_central}.}
\end{artitems}

\Articulo{Requisitos para cargos electivos}
\begin{artitems}
    \artitem{Para la \gls{jd}, véase el \aref{art:jd-requisitos} y \aref{art:jd-impedimentos} del presente \gls{estatuto}.}
    \artitem{Para el \gls{ce}, véanse el \aref{art:ce-requisitos} y el \aref{art:ce-impedimentos} del presente \gls{estatuto}.}
    \artitem{El \gls{ce} es responsable de verificar el cumplimiento de estos requisitos por parte de todos los postulantes, basado en la documentación especificada en el \gls{reglamento_electoral}.}
\end{artitems}

\Articulo{Inscripción de candidaturas}
\begin{artitems}
    \artitem{La postulación a la \gls{jd} se realiza mediante listas cerradas y bloqueadas, conforme a los cargos del \aref{art:jd-composicion}.}
    \artitem{La postulación al \gls{ce} es individual.}
    \artitem{El \gls{ce} es responsable de recibir y verificar las solicitudes de inscripción de candidaturas, asegurando que cumplan todos los requisitos y la documentación exigida por el \gls{estatuto} y el \gls{reglamento_electoral}.}
    \artitem{El procedimiento detallado, los plazos para la inscripción, la subsanación de observaciones y la publicación de las listas o candidaturas aptas serán definidos en el \gls{reglamento_electoral}. Las decisiones del \gls{ce} sobre las inscripciones deben ser justificadas y documentadas.}
\end{artitems}

\Articulo{Campaña electoral}
\begin{artitems}
    \artitem{El \gls{ce} establece el período oficial para la realización de la campaña electoral.}
    \artitem{El \gls{reglamento_electoral} norma las actividades permitidas y prohibidas durante la campaña, buscando garantizar la equidad, el respeto entre candidatos y el buen uso de los espacios y canales de comunicación del \gls{cefis}. El Reglamento puede incluir disposiciones sobre límites de gastos si la \gls{age} lo considera necesario.}
    \artitem{Los órganos del \gls{cefis} (\gls{jd}, \gls{jr}) y el propio \gls{ce} deben mantener estricta neutralidad durante la campaña.}
    \artitem{El \gls{ce} supervisa el cumplimiento de las normas de campaña y puede organizar debates o foros informativos.}
\end{artitems}

\Articulo{Miembros de mesa}
\begin{artitems}
    \artitem{Para cada mesa de sufragio (física o virtual), el \gls{ce} designa miembros de mesa (presidente, secretario, vocal) mediante el procedimiento que establezca el \gls{reglamento_electoral} (preferentemente sorteo público entre \glspl{agremiado_habilitado} no candidatos/as).}
    \artitem{El \gls{ce} es responsable de capacitar a los miembros de mesa sobre sus funciones, las cuales incluyen instalar la mesa, verificar la identidad de los votantes según el padrón, entregar el material de votación (o validar acceso virtual), custodiar el desarrollo del sufragio, realizar el escrutinio inicial y redactar el acta electoral correspondiente.}
    \artitem{Los miembros de mesa actúan bajo la dirección y supervisión del \gls{ce} y están protegidos en el ejercicio imparcial de sus funciones. El \gls{reglamento_electoral} establece los impedimentos para ser Miembro de Mesa.}
\end{artitems}

\Articulo{Jornada electoral}
\begin{artitems}
    \artitem{El \gls{ce} es responsable de organizar y supervisar la jornada electoral en la fecha y horario establecidos en la convocatoria.}
    \artitem{El \gls{reglamento_electoral} detalla el procedimiento de votación, sea presencial, virtual o mixto, garantizando la identificación del \gls{agremiado} votante, la emisión de voto único por elector y el secreto del mismo. Se implementan las medidas de seguridad técnica y logística necesarias.}
    \artitem{El \gls{ce} vela por el orden y la normalidad durante el desarrollo de la jornada.}
\end{artitems}

\Articulo{Escrutinio y actas electorales}
\begin{artitems}
    \artitem{Finalizada la votación, se realizará el escrutinio de votos de forma pública en cada mesa de sufragio, bajo la dirección de los miembros de mesa y la supervisión del \gls{ce}.}
    \artitem{El procedimiento para el conteo de votos, la resolución de votos observados o impugnados en mesa, y la elaboración del Acta Electoral detallando los resultados (votos por lista/candidato, votos blancos, nulos, total de votantes) es especificado en el \gls{reglamento_electoral}.}
    \artitem{Las Actas Electorales originales serán entregadas al \gls{ce} para la consolidación de resultados. Las copias serán entregadas a los personeros acreditados, si los hubiere.}
\end{artitems}

\Articulo{Consolidación y proclamación de resultados}
\begin{artitems}
    \artitem{El \gls{ce} es responsable de consolidar los resultados de todas las Actas Electorales, verificando su correcta sumatoria.}
    \artitem{Una vez resueltas las impugnaciones presentadas ante el \gls{ce} (conforme \aref{item:elecciones-impugnaciones-apelaciones:ce-1ra-instancia}) o vencido el plazo para ello, el \gls{ce} proclama oficialmente los resultados finales y a los candidatos electos.}
    \artitem{La proclamación se realiza mediante un Acta de Proclamación motivada, que será publicada en los canales oficiales del \gls{cefis} y archivada en el \gls{registro_central}.}
\end{artitems}

\Articulo[art:elecciones-impugnaciones-apelaciones]{Impugnaciones y apelaciones electorales}
\begin{artitems}
    \artitem[item:elecciones-impugnaciones-apelaciones:ce-1ra-instancia]{Sobre la impugnación ante el \gls{ce} en primera instancia:}
    \begin{enumerate}
        \subartitem{Cualquier \gls{agremiado} o lista participante puede impugnar actos específicos del proceso electoral, decisiones del \gls{ce} o los resultados consignados en las actas, dentro del plazo y por las causales establecidas en el \gls{reglamento_electoral}.}
        \subartitem{La impugnación debe ser presentada formalmente ante el \gls{ce}, debidamente sustentada y con los medios probatorios pertinentes.}
        \subartitem{El \gls{ce} resuelve la impugnación mediante resolución motivada y documentada en un plazo perentorio fijado por el \gls{reglamento_electoral}, notificando a las partes interesadas.}
    \end{enumerate}
    \artitem[item:elecciones-impugnaciones-apelaciones:agee-final-instancia]{Sobre la apelación ante la \gls{agee} en instancia final:}
    \begin{enumerate}
        \subartitem{Las resoluciones del \gls{ce} solo pueden ser apeladas ante la \gls{agee} convocada específicamente para tal fin (conforme \aref{item:agee-atribuciones:resolver-apelacion-ce}, \aref{item:ce-atribuciones:decisiones-efectivas}).}
        \subartitem{La apelación procederá únicamente en casos de grave irregularidad debidamente probada que, a criterio del apelante, vicie sustancialmente la validez del proceso o altere de forma determinante el resultado electoral. La carga de la prueba recae en el apelante.}
        \subartitem{El \gls{reglamento_electoral} establece el procedimiento y plazo para interponer la apelación ante la \gls{agee}.}
        \subartitem{La decisión de la \gls{agee} es adoptada por mayoría simple de los asistentes, debe ser motivada, se registra en el acta respectiva y tiene carácter definitivo e inapelable dentro del ámbito del \gls{cefis}.}
    \end{enumerate}
\end{artitems}
Las decisiones firmes del \gls{ce} (si no hay apelación) o de la \gls{agee} (en caso de apelación) son de cumplimiento obligatorio.

\Articulo{El Reglamento Electoral}
\begin{artitems}
    \artitem{Los procedimientos para la organización, ejecución y control de los procesos electorales se establecen en el \gls{reglamento_electoral} del \gls{cefis}.}
    \artitem{El \gls{reglamento_electoral} es elaborado o actualizado por el \gls{ce} y debe ser aprobado o modificado por la \gls{age} por mayoría simple.}
    \artitem{El \gls{reglamento_electoral} debe ser coherente con los principios y disposiciones del presente \gls{estatuto}.}
    \artitem{El \gls{ce} está obligado a aplicar el \gls{reglamento_electoral} vigente y puede proponer su actualización a la \gls{age} basándose en la experiencia de procesos anteriores.}
    \artitem[item:reglamento-electoral:revision-age-bienal]{La \gls{age} revisa formalmente el \gls{reglamento_electoral} al menos cada dos (2) años (\aref{item:ageo-atribuciones:actualizar-criterios-bienal}).}
\end{artitems}

\Titulo{De las Comisiones de Trabajo}

\Articulo[art:comisiones-naturaleza-creacion]{Naturaleza y creación}
Las \glspl{comisiones_trabajo} son órganos de apoyo, de carácter temporal o permanente, destinadas a estudiar, proponer o ejecutar tareas específicas que contribuyan a los fines del \gls{cefis}. Pueden ser creadas por acuerdo de la \gls{age} o de la \gls{jd}. El acuerdo de creación debe especificar, como mínimo:
\begin{artitems}
    \artitem{El mandato específico y los objetivos de la comisión.}
    \artitem{Su duración estimada o carácter permanente.}
    \artitem{Su composición inicial o el mecanismo para designar a sus miembros.}
    \artitem{El órgano (\gls{age} o \gls{jd}) o la Secretaría de la \gls{jd} responsable de su supervisión directa.}
\end{artitems}
Este acuerdo se registra en el acta correspondiente (de \gls{age} o \gls{jd}) y una copia se archiva en el \gls{registro_central}.

\Articulo{Mandato y funciones}
\begin{artitems}
    \artitem{Las funciones y el ámbito de acción de cada \gls{comisiones_trabajo} están estrictamente delimitados por el mandato establecido en su acuerdo de creación.}
    \artitem{En ningún caso podrán arrogarse funciones decisorias o de representación que correspondan a los órganos de gobierno (\aref{art:organos-gobierno}). Sus conclusiones o propuestas tienen carácter de recomendación para el órgano que la creó o supervisa.}
\end{artitems}

\Articulo{Composición y coordinación}
\begin{artitems}
    \artitem{Los miembros son designados por el órgano que las creó (\gls{age} o \gls{jd}), procurando la participación voluntaria de \glspl{agremiado} con interés, conocimiento o experiencia en la materia específica.}
    \artitem{Cada comisión cuenta con un/a Coordinador/a, quien es elegido democráticamente entre sus miembros o designado directamente en el acuerdo de creación.}
    \artitem{El/la Coordinador/a es responsable de organizar el trabajo interno, dirigir las reuniones, asegurar el cumplimiento del mandato y servir como enlace principal con el órgano responsable de su supervisión.}
\end{artitems}

\Articulo{Funcionamiento y reporte}
\begin{artitems}
    \artitem{Las comisiones establecen su propio ritmo y método de trabajo para cumplir su mandato, manteniendo comunicación con el órgano supervisor.}
    \artitem{Las comisiones llevan un registro simple y ordenado de sus actividades principales, discusiones y acuerdos internos, que está a disposición del órgano supervisor.}
    \artitem{El/la Coordinador/a presenta informes periódicos de avance al órgano que creó la comisión o al responsable de su supervisión, con la frecuencia que este determine en el acuerdo de creación o posteriormente.}
    \artitem{Los informes finales o aquellos considerados de especial importancia por el órgano supervisor pueden ser puestos en conocimiento de la \gls{jr} y/o publicados a través de los canales oficiales del \gls{cefis}.}
\end{artitems}

\Articulo{Disolución}
\begin{artitems}
    \artitem{Las \glspl{comisiones_trabajo} se disuelven automáticamente al cumplir su mandato específico o al vencer el plazo establecido en su acuerdo de creación.}
    \artitem{También pueden ser disueltas antes del plazo por acuerdo documentado del órgano que las creó, si se considera que han cumplido su objetivo, que su continuación ya no es necesaria, o por incumplimiento grave de su mandato.}
    \artitem{Al momento de su disolución, o al finalizar un periodo significativo si es permanente, la comisión debe presentar un informe final documentado de sus actividades, logros y recomendaciones al órgano que la creó.}
    \artitem{Tanto el informe final como el acuerdo de disolución (si aplica) se archivan en el \gls{registro_central}.}
\end{artitems}

\Titulo{Conducta, procedimientos disciplinarios y apelaciones}

\Articulo{Conducta esperada}
Todo/a \gls{agremiado} debe actuar conforme a los principios (\aref{art:principios-cefis}) y deberes (\aref{art:deberes-agremiados}). Se espera una conducta basada en el respeto mutuo, la probidad, la colaboración, el uso adecuado de los recursos y bienes del \gls{cefis}, y la integridad en el desempeño de cualquier cargo o representación.

\Articulo[art:faltas-definicion]{Definición de faltas}
Para efectos de este \gls{estatuto}, las faltas contra los deberes, principios o normas del \gls{cefis} se clasifican en:
\begin{artitems}
    \artitem{\Gls{falta_leve}: Incumplimientos menores de deberes o normativas internas que no causen perjuicio significativo al \gls{cefis}, sus fines o sus \glspl{agremiado} (Ej: negligencia puntual en tareas asignadas sin consecuencias graves).}
    \artitem{\Gls{falta_grave}: Acciones u omisiones que atentan seriamente contra los principios, fines o patrimonio del \gls{cefis}, los derechos de otros/as \glspl{agremiado}, o el normal funcionamiento de sus órganos (Ej: incumplimiento reiterado de deberes, mal uso comprobado de fondos o bienes del \gls{cefis}, actos de hostigamiento o discriminación, obstrucción deliberada de acuerdos de \gls{age}, incumplimiento grave del \gls{estatuto} o reglamentos).}
    \artitem{\Gls{falta_muy_grave}: \glspl{falta_grave} que, por su naturaleza, intencionalidad, daño causado o reincidencia, revisten una especial trascendencia negativa para el \gls{cefis} o sus \glspl{agremiado} (Ej: apropiación indebida de fondos, actos de violencia física o acoso sexual comprobados, falsificación de documentos oficiales del \gls{cefis}, usurpación de funciones).}
\end{artitems}
La calificación específica de una falta se determinará al concluir el procedimiento correspondiente, basándose en los hechos probados y la normativa aplicable.

\Articulo{Mecanismo de denuncia}
\begin{artitems}
    \artitem{Cualquier \gls{agremiado} que tenga conocimiento de una presunta falta puede presentar una denuncia formal.}
    \artitem{La denuncia sobre presuntas \glspl{falta_grave} o \glspl{falta_muy_grave} se presenta por escrito (físico o correo electrónico oficial), debidamente sustentada y dirigida a la Coordinación de la \gls{jr}. La Coordinación registra la recepción de la denuncia.}
    \artitem{La denuncia sobre presuntas \glspl{falta_leve} puede presentarse ante la Secretaría General de la \gls{jd} o ante la Coordinación de la \gls{jr}.}
    \artitem{La persona denunciante puede solicitar la reserva de su identidad, la cual es mantenida por la \gls{jr} o \gls{jd} durante la investigación preliminar hasta donde sea procesalmente posible y no impida el derecho a la defensa.}
    \artitem[item:denuncia-mecanismo:proteccion-buena-fe]{Los/as \glspl{agremiado} que presenten denuncias de buena fe están protegidos contra represalias, conforme a la supervisión de la \gls{jr} (\aref{item:derechos-agremiados:proteccion-denuncias}, \aref{item:jr-atribuciones:velar-proteccion-represalias}).}
\end{artitems}

\Articulo[art:investigacion-preliminar-proceso]{Investigación preliminar}
\begin{artitems}
    \artitem{\Glspl{falta_leve}: Son gestionadas directamente por la \gls{jd} o la \gls{jr} mediante comunicación escrita o amonestación verbal, buscando la corrección de la conducta. Se deja constancia simple si se considera necesario.}
    \artitem[item:art:investigacion-preliminar-proceso:faltas-graves-muy-graves]{\Glspl{falta_grave} o \glspl{falta_muy_grave} (procedimiento interno):}
    \begin{enumerate}
        \subartitem{Recibida una denuncia, la \gls{jr} inicia una investigación preliminar (\aref{item:jr-atribuciones:investigacion-preliminar-faltas}). Para casos de alta complejidad o que involucren a miembros de la propia \gls{jr}, esta puede solicitar a la \gls{age} la conformación de una comisión investigadora ad hoc.}
        \subartitem[subitem:investigacion-preliminar-proceso:alcance]{La investigación preliminar tiene como objetivo recabar información y elementos de juicio sobre los hechos denunciados. Incluye, como mínimo, la recopilación de documentos o testimonios pertinentes y la oportunidad para que el/la \gls{agremiado} investigado/a presente sus descargos iniciales por escrito. Se desarrolla con la debida reserva.}
        \subartitem{La \gls{jr} (o la comisión ad hoc) elabora un informe documentado en un plazo razonable (no mayor a veinte (20) días hábiles, prorrogable justificadamente por la \gls{jr}). Dicho informe contiene los hechos indagados, los descargos recibidos, las conclusiones preliminares y una recomendación sobre si procede archivar la denuncia o iniciar un procedimiento sancionador ante la \gls{age}.}
        \subartitem{El informe es registrado y elevado a la \gls{jd} para que convoque a \gls{agee} si la recomendación es iniciar procedimiento sancionador. Si la recomendación es archivar, se notifica al denunciante (si es identificable) y al/la investigado/a.}
    \end{enumerate}
    \artitem{Actuación ante hechos que podrían constituir delito (violencia, acoso sexual, hurto, etc.):}
    \begin{enumerate}
        \subartitem{El \gls{cefis}, a través de la \gls{jd} o la Secretaría de Bienestar y Género, tiene el deber de informar al/la \gls{agremiado} afectado/a sobre sus derechos y las vías de denuncia ante las autoridades universitarias (Secretaría de Instrucción, Defensoría Universitaria) y/o externas (Policía Nacional, Ministerio Público). Ofrece orientación y acompañamiento si el/la afectado/a lo solicita.}
        \subartitem{La decisión de denunciar ante instancias externas corresponde exclusivamente al/la afectado/a o a las autoridades competentes según ley. El \gls{cefis} respeta esta decisión.}
        \subartitem{Si se inicia un proceso formal de investigación o judicial por estos hechos ante autoridades externas competentes, y el/la \gls{agremiado} denunciado/a ocupa un cargo en el \gls{cefis} (\gls{jd}, \gls{jr}, \gls{ce}, \glspl{delegatura}, \glspl{subdelegatura}) o su permanencia activa representa un riesgo evidente para la comunidad estudiantil, la \gls{jd}, previo informe a la \gls{jr}, puede acordar la suspensión cautelar del/la \gls{agremiado} de sus cargos y/o participación en actividades del \gls{cefis}, mientras dure la investigación externa. Esta medida es administrativa, no disciplinaria, y debe ser comunicada formalmente y registrada.}
        \subartitem{Independientemente de la vía externa, si los hechos también constituyen una \gls{falta_grave} o \gls{falta_muy_grave} según el \aref{art:faltas-definicion}, la \gls{jr} puede realizar la investigación preliminar (\aref{item:art:investigacion-preliminar-proceso:faltas-graves-muy-graves}) y, si el/la afectado/a consiente expresamente en seguir también la vía interna del \gls{cefis} o si la falta afecta directamente al gremio (ej.\ malversación), se puede continuar con el procedimiento sancionador ante la \gls{age} (\aref{art:age-procedimiento-sancionador}). Las sanciones internas son independientes de las externas.}
    \end{enumerate}
\end{artitems}

\Articulo[art:age-procedimiento-sancionador]{Procedimiento sancionador ante la AGE (faltas graves/muy graves)}
\begin{artitems}
    \artitem{Recibido el informe de la \gls{jr}, o comisión ad hoc, que recomienda iniciar procedimiento sancionador, la \gls{jd} convoca a \gls{agee} cuyo punto de agenda único o principal será resolver dicho procedimiento.}
    \artitem{El/la \gls{agremiado} investigado/a es notificado formalmente por la \gls{jd} (con copia a la \gls{jr}) de la convocatoria y del informe de investigación, con una antelación mínima de cinco (5) días hábiles a la fecha de la \gls{agee}.}
    \artitem{Durante la \gls{agee}, se garantiza el debido proceso, que incluye:}
    \begin{enumerate}
        \subartitem{Presentación del caso por parte de un miembro designado de la \gls{jr} o comisión ad hoc.}
        \subartitem{Derecho del \gls{agremiado} investigado/a a ejercer su defensa personalmente (o con asistencia de otro \gls{agremiado} si lo solicita), presentando sus descargos, pruebas y argumentos.}
        \subartitem{Posibilidad de un debate ordenado dirigido por quien presida la \gls{age}.}
    \end{enumerate}
    \artitem{Concluida la presentación y el debate, la \gls{agee} delibera y vota la aplicación o no de una sanción. La decisión sobre la existencia de la falta y la sanción a imponer debe ser adoptada por la mayoría según \aref{art:age-acuerdos-mayorias} y \aref{art:age-sanciones-aplicables}.}
    \artitem{La decisión final debe ser motivada, consta en el acta de la \gls{agee} y es notificada formalmente al \gls{agremiado} sancionado/a y al denunciante (si es identificable).}
\end{artitems}

\Articulo[art:age-sanciones-aplicables]{Sanciones aplicables por AGE}
Según la gravedad de la falta probada, la \gls{age} puede imponer las siguientes sanciones:
\begin{artitems}
    \artitem{Amonestación escrita: Llamada de atención formal registrada en acta de \gls{age}. Aplicable a \glspl{falta_grave}. (Mayoría simple).}
    \artitem{Suspensión temporal de derechos: Inhabilitación para participar en actividades, votar o ser elegido/a por un período determinado (máximo 1 año). Aplicable a \glspl{falta_grave} o \glspl{falta_muy_grave}. (Mayoría simple).}
    \artitem[item:age-sanciones-aplicables:remocion-cargo]{Remoción del cargo: Destitución de cualquier cargo electivo o de representación dentro del \gls{cefis}. Aplicable a \glspl{falta_grave} o \glspl{falta_muy_grave} cometidas en ejercicio del cargo. (Mayoría calificada 2/3, conforme \aref{subitem:age-acuerdos-mayorias:remocion-jd-2tercios}).}
    \artitem[item:age-sanciones-aplicables:exclusion-agremiado]{Exclusión del \gls{cefis}: Pérdida definitiva de la condición de \gls{agremiado}. Aplicable únicamente a \glspl{falta_muy_grave}. (Mayoría calificada 2/3, conforme \aref{subitem:age-acuerdos-mayorias:exclusion-agremiado-2tercios}).}
\end{artitems}
Las sanciones se aplican respetando el principio de proporcionalidad entre la falta cometida y la sanción impuesta. La reincidencia puede ser considerada un agravante.

\Articulo[art:sanciones-registro-confidencial]{Registro confidencial de sanciones}
\begin{artitems}
    \artitem{Todas las sanciones firmes impuestas por la \gls{age} se registran en archivo específico de carácter confidencial, gestionado por la Secretaría de Actas y Economía como parte del \gls{registro_central}.}
    \artitem{Este archivo consigna los datos del \gls{agremiado} sancionado/a, la falta cometida, la sanción impuesta, la fecha y el órgano que la decidió.}
    \artitem{El acceso a este archivo está restringido a la persona titular de la Secretaría de Actas y Economía y de la Secretaría General para fines de gestión interna, y a la \gls{jr} o al \gls{ce} para verificar antecedentes en procesos disciplinarios o electorales, previa solicitud justificada y registrada. Su contenido no es de acceso público general.}
\end{artitems}

\Articulo[art:apelaciones-no-disciplinarias-generales]{Apelaciones generales (no disciplinarias graves/no electorales)}
\begin{artitems}
    \artitem{Las decisiones administrativas adoptadas por la \gls{jd}, sus Secretarías o las \glspl{comisiones_trabajo}, que no constituyan sanciones por \glspl{falta_grave} o \glspl{falta_muy_grave} (\aref{art:age-procedimiento-sancionador}) ni decisiones del \gls{ce} (\aref{art:elecciones-impugnaciones-apelaciones}), pueden ser apeladas por cualquier \gls{agremiado} afectado/a.}
    \artitem{La apelación se presenta por escrito y de forma motivada ante la Coordinación de la \gls{jr} dentro de los diez (10) días hábiles siguientes a la comunicación de la decisión.}
    \artitem[item:apelaciones-no-disciplinarias-generales:revision-jr]{La \gls{jr} evalúa la apelación, solicita informes que considere necesarios al órgano que emitió la decisión, y emite una recomendación documentada a la \gls{jd} o a la \gls{age}, según la trascendencia del asunto, en un plazo de quince (15) días hábiles.}
    \artitem[item:apelaciones-no-disciplinarias-generales:elevacion-age]{Si la decisión apelada fue de la \gls{jd} y esta no acoge la recomendación de la \gls{jr}, o si el asunto es de competencia originaria de la \gls{age}, la \gls{jr} puede elevar el caso a la siguiente \gls{age} (ordinaria o extraordinaria) para su resolución final. La decisión de la \gls{age} es definitiva en el ámbito del \gls{cefis} y se registra en acta.}
\end{artitems}

\Articulo[art:anulacion-razones-eticas]{Anulación por razones éticas}
\begin{artitems}
    \artitem{Si una decisión o acto de un órgano del \gls{cefis} formalmente correcto respecto al \gls{estatuto} o reglamentos produce un resultado manifiestamente injusto, contrario a los principios fundamentales del \gls{cefis} (\aref{art:principios-cefis}), o perjudicial para la integridad de la comunidad estudiantil, cualquier \gls{agremiado}, la \gls{jd} o la \gls{jr} pueden solicitar su revisión a la \gls{agee}.}
    \artitem{La solicitud debe presentarse por escrito, fundamentando de manera sólida y clara la contravención ética o la injusticia manifiesta del resultado.}
    \artitem[item:anulacion-razones-eticas:procedimiento-agee]{La \gls{agee} evalúa la solicitud y los antecedentes del caso. Si considera justificada la revisión por razones éticas, puede, por mayoría calificada de dos tercios (2/3) de los \glspl{agremiado} presentes y votantes (conforme \aref{subitem:age-acuerdos-mayorias:anulacion-etica-2tercios}), confirmar, modificar o anular la decisión o acto original.}
    \artitem{La decisión de la \gls{agee} debe ser explícitamente motivada en consideraciones éticas y de justicia, quedando registrada en acta. Este mecanismo es de uso excepcional.}
\end{artitems}

\Articulo[art:revisiones-posincidente]{Revisiones posincidente}
\begin{artitems}
    \artitem{Tras la conclusión de un procedimiento sancionador por \gls{falta_grave} o \gls{falta_muy_grave}, o después de cualquier incidente significativo que haya requerido una intervención excepcional (como una anulación por razones éticas o la resolución de una crisis interna), la \gls{jr} tendrá la responsabilidad de conducir o supervisar una revisión del caso.}
    \artitem{Esta revisión no tiene como fin reevaluar la responsabilidad individual ya determinada, sino identificar posibles fallas en los procedimientos, normativas, comunicación interna o salvaguardas del \gls{cefis} que pudieron haber contribuido al incidente.}
    \artitem{La \gls{jr} elabora un informe documentado con enfoque sistémico, que incluye recomendaciones concretas para mejorar los procesos, reglamentos o prácticas del \gls{cefis} y prevenir la recurrencia de situaciones similares.}
    \artitem{Este informe es presentado a la \gls{jd} y a la \gls{age}, y sus recomendaciones serán consideradas para futuras actualizaciones normativas o de gestión. El informe se archiva en el \gls{registro_central}.}
\end{artitems}

\Titulo{Modificación del Estatuto}

\Articulo{Iniciativa de modificación}
La propuesta para modificar parcial o totalmente el presente \gls{estatuto} puede ser presentada formalmente por:
\begin{artitems}
    \artitem{La \gls{jd}, mediante acuerdo registrado en acta.}
    \artitem{La \gls{jr}, mediante acuerdo registrado en acta.}
    \artitem[item:estatuto-iniciativa-modificacion:por-agremiados-10pct]{Un número no menor al diez por ciento (10\%) de los \glspl{agremiado_habilitado}, mediante solicitud escrita fundamentada dirigida a la \gls{jd}, la cual registra su recepción.}
\end{artitems}

\Articulo{Procedimiento de modificación}
\begin{artitems}
    \artitem{La propuesta de modificación, sea cual sea su origen, debe ser presentada por escrito, especificando los artículos a modificar, suprimir o añadir, y adjuntando una exposición de motivos que justifique el cambio propuesto.}
    \artitem{La \gls{jd} es responsable de verificar que la propuesta cumpla los requisitos formales y, en caso de iniciativa de los \glspl{agremiado} (\aref{item:estatuto-iniciativa-modificacion:por-agremiados-10pct}), de incluirla obligatoriamente en la agenda de la próxima \gls{agee} convocada para tal efecto o en una \gls{agee} específica.}
    \artitem{La convocatoria a la \gls{agee} que incluya la modificación del \gls{estatuto} debe realizarse con una anticipación mínima de quince (15) días calendario. La convocatoria debe adjuntar el texto completo de la propuesta de modificación y su exposición de motivos.}
    \artitem{La difusión de la convocatoria y la propuesta se realiza a través de los canales oficiales de comunicación del \gls{cefis}.}
    \artitem{Durante la \gls{agee} se realiza el debate correspondiente sobre la propuesta de modificación.}
\end{artitems}

\Articulo{Aprobación de la modificación}
\begin{artitems}
    \artitem{Para la instalación válida de la \gls{agee} que trate la modificación del \gls{estatuto}, se aplican las reglas de quórum establecidas (\aref{art:age-quorum-1ra-llamada}, \aref{art:age-quorum-2da-llamada} y \aref{art:age-quorum-2da-convocatoria}).}
    \artitem[item:estatuto-aprobacion-modificacion:mayoria-2tercios]{La aprobación de cualquier modificación al presente \gls{estatuto} requiere el voto favorable de una mayoría de dos tercios (2/3) de \glspl{agremiado_habilitado} presentes y votantes en la \gls{agee} (\aref{subitem:age-acuerdos-mayorias:modificar-estatuto-2tercios}, \aref{item:agee-atribuciones:modificar-estatuto}). La votación es registrada en acta.}
\end{artitems}

\Articulo{Registro y vigencia}
\begin{artitems}
    \artitem{La modificación aprobada del \gls{estatuto} será transcrita literalmente en el acta de la \gls{agee} correspondiente.}
    \artitem{El texto íntegro del \gls{estatuto} actualizado, incorporando las modificaciones aprobadas, es elaborado por la Secretaría de Actas y Economía bajo supervisión de la \gls{jd}, y archivado formalmente en el \gls{registro_central}.}
    \artitem{La \gls{jd} es responsable de publicar el \gls{estatuto} actualizado a través de los canales oficiales en un plazo no mayor a diez (10) días hábiles desde su aprobación.}
    \artitem{Las modificaciones entran en vigencia a partir del día siguiente de su publicación oficial, salvo que la propia \gls{agee} acuerde una fecha distinta, la cual consta en acta.}
\end{artitems}

\Articulo{Revisión periódica integral}
\begin{artitems}
    \artitem{Independientemente de las modificaciones puntuales que puedan surgir, el presente \gls{estatuto} debe ser sometido a una revisión integral al menos cada cuatro (4) años.}
    \artitem{La \gls{ageo} designa una Comisión Revisora del \gls{estatuto}, a propuesta de la \gls{jd} o la \gls{jr}, con el mandato específico de evaluar la vigencia, coherencia y efectividad del \gls{estatuto} y proponer las actualizaciones que considere necesarias.}
    \artitem{La Comisión Revisora presenta sus conclusiones y propuestas de modificación documentadas, las cuales siguen el procedimiento regular establecido en este Título para su debate y aprobación por la \gls{agee}.}
\end{artitems}

\Titulo{Disolución y liquidación}

\Articulo{Causales de disolución}
El \gls{cefis} solo podrá ser disuelto por las siguientes causales:
\begin{artitems}
    \artitem{Por decisión expresa de la \gls{agee}, adoptada conforme al procedimiento establecido en este Título.}
    \artitem{Por imposibilidad manifiesta y permanente de cumplir con los fines y objetivos establecidos en el \aref{art:fines-obj-cefis} del presente \gls{estatuto}, declarada formalmente por la \gls{agee}.}
\end{artitems}

\Articulo{Procedimiento de disolución}
\begin{artitems}
    \artitem{La propuesta de disolución del \gls{cefis} solo puede ser presentada por la \gls{jd}, la \gls{jr}, o por un número no menor al veinte por ciento (20\%) de los \glspl{agremiado_habilitado}, mediante solicitud escrita y fundamentada dirigida a la \gls{jd}.}
    \artitem{Recibida una propuesta válida, la \gls{jd} convoca obligatoriamente a una \gls{agee} con punto de agenda único: ``Debate y decisión sobre la disolución del \gls{cefis}''.}
    \artitem{La convocatoria debe realizarse con una anticipación mínima de veinte (20) días calendario, adjuntando la propuesta de disolución y su fundamentación. La difusión es máxima a través de todos los canales oficiales.}
    \artitem{Para la instalación válida de esta \gls{agee} se requiere un quórum especial de asistencia de, al menos, el treinta por ciento (30\%) de los \glspl{agremiado_habilitado} en primera convocatoria. En segunda convocatoria, a realizarse una hora después, se instala con la asistencia de, al menos, el veinte por ciento (20\%) de los \glspl{agremiado_habilitado}. Si no se alcanza el quórum en segunda convocatoria, no se puede tratar la disolución y la propuesta queda sin efecto.}
    \artitem[item:disolucion-procedimiento:voto-2tercios]{La decisión de disolver el \gls{cefis} requiere el voto favorable de una mayoría de dos tercios (2/3) de \glspl{agremiado_habilitado} presentes y votantes en la \gls{agee} válidamente instalada (\aref{subitem:age-acuerdos-mayorias:disolucion-cefis-2tercios}, \aref{item:agee-atribuciones:disolver-cefis}). La votación será nominal y registrada en acta.}
\end{artitems}

\Articulo{Comisión Liquidadora}
\begin{artitems}
    \artitem{Aprobada la disolución, la misma \gls{agee} designa una \gls{comision_liquidadora} compuesta por tres (3) personas \glspl{agremiado} titulares y una (1) suplente, quienes no pueden ser miembros de la última \gls{jd}.}
    \artitem{La \gls{comision_liquidadora} asume la representación del \gls{cefis} exclusivamente para los fines de la liquidación y es responsable de:}
    \begin{enumerate}
        \subartitem{Realizar el inventario final de todos los bienes y activos del \gls{cefis}.}
        \subartitem{Realizar el balance final y determinar el activo y pasivo existente.}
        \subartitem{Cobrar los créditos pendientes a favor del \gls{cefis}.}
        \subartitem{Pagar todas las deudas y obligaciones pendientes del \gls{cefis}.}
        \subartitem{Distribuir el patrimonio remanente conforme a lo establecido en el \aref{art:destino-patrimonio-remanente}.}
    \end{enumerate}
    \artitem{La \gls{comision_liquidadora} actúa bajo supervisión de la propia \gls{agee}, a la cual presenta un informe final documentado de todo el proceso de liquidación para su aprobación. El mandato de la comisión cesa con la aprobación de dicho informe.}
    \artitem{Las personas miembros de la \gls{comision_liquidadora} son responsables personal y solidariamente por los actos realizados en el ejercicio de sus funciones que excedan su mandato o contravengan el \gls{estatuto} o los acuerdos de la \gls{age}.}
\end{artitems}

\Articulo[art:destino-patrimonio-remanente]{Destino del patrimonio remanente}
\begin{artitems}
    \artitem{Una vez pagadas todas las deudas y obligaciones del \gls{cefis}, el patrimonio remanente (bienes muebles, saldos financieros, etc.) no puede ser distribuido, bajo ninguna circunstancia, entre los/as \glspl{agremiado}.}
    \artitem{El patrimonio remanente es destinado íntegramente, por decisión de la misma \gls{agee} que aprueba la disolución y a propuesta documentada de la \gls{comision_liquidadora}, a una o más de las siguientes opciones, priorizando siempre fines educativos, culturales o de bienestar estudiantil dentro de la \gls{unmsm}:}
    \begin{enumerate}
        \subartitem{La Biblioteca de la Facultad de Ciencias Físicas.}
        \subartitem{Laboratorios o proyectos de investigación de la \gls{epf}.}
        \subartitem{Programas de bienestar estudiantil administrados por la Facultad o la Universidad.}
        \subartitem{Otro gremio estudiantil formalmente reconocido dentro de la Facultad o la Universidad con fines similares a los del \gls{cefis}.}
    \end{enumerate}
    \artitem{La decisión sobre el destino específico del patrimonio remanente consta en el acta final de la \gls{agee} que apruebe el informe de la \gls{comision_liquidadora}.}
\end{artitems}

\Articulo{Registro de la disolución y liquidación}
\begin{artitems}
    \artitem{El acuerdo de disolución, la designación de la \gls{comision_liquidadora}, el destino del patrimonio remanente y la aprobación del informe final de liquidación constan en las actas respectivas de la \gls{agee}.}
    \artitem{Estos documentos, junto con el inventario y balance finales, serán: (a) archivados permanentemente en el \gls{registro_central} o entregados a la Facultad de Ciencias Físicas para su custodia; y (b) compilados en formato digital para su descarga íntegra por los/as \glspl{agremiado} interesados/as durante un período mínimo de un (1) año posterior a la disolución.}%
    \artitem{La \gls{jd} saliente o la \gls{comision_liquidadora} comunica formalmente la disolución del \gls{cefis} a las autoridades de la \gls{epf} y de la Facultad de Ciencias Físicas.}
\end{artitems}

\Titulo{Disposiciones finales y transitorias}

\Articulo{Entrada en vigencia}
El presente \gls{estatuto} entra en vigencia al día siguiente de su publicación oficial por parte de la \gls{jd} a través de los canales oficiales de comunicación del \gls{cefis}, previa aprobación por \gls{age} conforme a la normativa anterior vigente o, si no existiera, por mayoría simple en Asamblea convocada para tal fin. La aprobación consta en acta.

\Articulo{Normas supletorias}
En todo lo no previsto expresamente en el presente \gls{estatuto}, regirán supletoriamente las disposiciones del Estatuto de la \gls{unmsm}, la Ley Universitaria N 30220 y sus modificatorias, la Constitución Política del Perú, y las normas pertinentes del Código Civil en materia de asociaciones, siempre que no contravengan los principios y fines del \gls{cefis}.

\Articulo{Interpretación del Estatuto}
\begin{artitems}
    \artitem[item:estatuto-interpretacion:final-age]{La interpretación auténtica y definitiva de las disposiciones del presente \gls{estatuto} corresponde exclusivamente a la \gls{age}, adoptada por mayoría simple en sesión ordinaria o extraordinaria.}
    \artitem{La \gls{jd} o la \gls{jr} podrán emitir opiniones interpretativas documentadas sobre la aplicación del \gls{estatuto} para guiar la gestión ordinaria, las cuales podrán ser sometidas a ratificación o modificación por la \gls{age} si se genera controversia o a solicitud de un órgano o un número significativo de \glspl{agremiado}. Dichas opiniones son archivadas en el \gls{registro_central}.}
\end{artitems}

\Articulo{Disposición transitoria}
\begin{artitems}
    \artitem{Todos los reglamentos internos del \gls{cefis} existentes a la fecha de entrada en vigencia de este \gls{estatuto} (incluyendo, pero no limitándose al \gls{reglamento_electoral} y \gls{reglamento_asambleas}) deben ser revisados y adecuados a este \gls{estatuto}.}
    \artitem{La \gls{jd}, la \gls{jr} y el \gls{ce} (según corresponda a cada reglamento) son responsables de presentar las propuestas de adecuación a la \gls{age} para su aprobación en un plazo máximo de noventa (90) días calendario contados desde la entrada en vigencia del presente \gls{estatuto}.}
    \artitem{Hasta que no se aprueben los reglamentos adecuados, seguirán aplicándose los anteriores en lo que no contradigan expresamente a este \gls{estatuto}, bajo la interpretación del órgano competente (\gls{jd}, \gls{jr}, \gls{ce}) supervisada por la \gls{age} si fuera necesario.}
\end{artitems}

\Articulo{Primera elección bajo este Estatuto}
La primera elección de la \gls{jd} y del \gls{ce} bajo la vigencia del presente \gls{estatuto} se realizará conforme a las disposiciones del Título V y es convocada y organizada por un \gls{ce} Ad-Hoc designado por la misma Asamblea General que apruebe este \gls{estatuto}. El proceso electoral completo deberá concluir en un plazo no mayor a sesenta (60) días calendario desde la entrada en vigencia del \gls{estatuto}.

\Articulo{Derogación}
A partir de la entrada en vigencia del presente \gls{estatuto}, quedan derogados todos los estatutos, reglamentos y disposiciones anteriores del \gls{cefis} que se le opongan total o parcialmente.

\clearpage
\printglossary[type=main, title={Glosario de términos}]

\end{document}
